\section{Rangga Putra Ramdhani}
\subsubsection{Pemahanan Teori}
\begin{enumerate}
    \item Apa itu fungsi, inputan fungsi dan kembalian fungsi dengan contoh kode program
    lainnya.
    Fungsi adalah bagian dari program yang dapat digunakan ulang.
    Berikut merupakan contoh fungsi dan cara pemanggilannya
    \lstinputlisting[firstline=124, lastline=127]{src/1174056_praktek.py}

    Fungsi dapat membaca parameter, parameter adalah nilai yang disediakan kepada fungsi, dimana nilai ini akan menentukan output yang akan dihasilkan fungsi.
    \lstinputlisting[firstline=129, lastline=132]{src/1174056_praktek.py}

    Statemen return digunakan untuk keluar dari fungsi. Kita juga dapat menspesifikasikan nilai kembalian.
    \lstinputlisting[firstline=134, lastline=141]{src/1174056_praktek.py}

    \item Apa itu paket dan cara pemanggilan paket atau library dengan contoh kode
    program lainnya.
    Untuk memudahkan dalam pemanggilan fungsi yang di butuhkan, agar dapat dipanggil berulang.
    Cara pemanggilannya
    \lstinputlisting[firstline=143, lastline=144]{src/1174056_praktek.py}

    \item Jelaskan Apa itu kelas, apa itu objek, apa itu atribut, apa itu method dan
    contoh kode program lainnya masing-masing.
    kelas merupakan sebuah blueprint yang mepresentasikan objek.
    objek adalah hasil cetakan dadri sebuah kelas.
    method adalah suatu upaya yang digunakan oleh object.
    \lstinputlisting[firstline=146, lastline=168]{src/1174056_praktek.py}

    \item Jelaskan cara pemanggikan library kelas dari instansiasi dan pemakaiannya den-
    gan contoh program lainnya.
    Cara Pemanggilanya 
    \begin{itemize}
        \item pertama import terlebih dahulu filenya.
        \item kemudian buat variabel untuk menampung datanya
        \item setelah itu panggil nama classnya dan panggil methodnya
        \item Gunakan perintah print untuk menampilkan hasilnya

    \end{itemize}
    \lstinputlisting[firstline=170, lastline=175]{src/1174056_praktek.py}

    \item Jelaskan dengan contoh pemakaian paket dengan perintah from kalkulator im-
    port Penambahan disertai dengan contoh kode lainnya.
    Penggunaan paket from namafile import, itu berfungsi untuk memanggil file dan fungsinya
    \lstinputlisting[firstline=143, lastline=144]{src/1174056_praktek.py}

    \item Jelaskan dengan contoh kodenya, pemakaian paket fungsi apabila le library
    ada di dalam folder.
    Pemakaian paket adalah perkumpulan fungsi-fungsi. contoh kodenya adalah sebagai berikut :

    \item Jelaskan dengan contoh kodenya, pemakaian paket kelas apabila le library ada
    di dalam folder.
    \lstinputlisting[firstline=184, lastline=184]{src/1174056_praktek.py}

\end{enumerate}
\subsubsection{Ketrampilan Pemrograman}
\begin{enumerate}
    \item Buatlah fungsi dengan inputan variabel NPM, dan melakukan print luaran huruf
    yang dirangkai dari tanda bintang, pagar atau plus dari NPM kita. Tanda
    bintang untuk NPM mod 3=0, tanda pagar untuk NPM mod 3 =1, tanda plus
    untuk NPM mod3=2.
    \lstinputlisting[firstline=184, lastline=234]{src/1174056_praktek.py}

    \item Buatlah fungsi dengan inputan variabel berupa NPM. kemudian dengan meng-
    gunakan perulangan mengeluarkan print output sebanyak dua dijit belakang
    NPM.
    \lstinputlisting[firstline=237, lastline=243]{src/1174056_praktek.py}

    \item Buatlah fungsi dengan dengan input variabel string bernama NPM dan beri
    luaran output dengan perulangan berupa tiga karakter belakang dari NPM se-
    banyak penjumlahan tiga dijit tersebut.
    \lstinputlisting[firstline=245, lastline=255]{src/1174056_praktek.py}

    \item Buatlah fungsi hello word dengan input variabel string bernama NPM dan
    beri luaran output berupa digit ketiga dari belakang dari variabel NPM meng-
    gunakan akses langsung manipulasi string pada baris ketiga dari variabel NPM.
    \lstinputlisting[firstline=257, lastline=263]{src/1174056_praktek.py}

    \item buat fungsi program dengan input variabel NPM dan melakukan print nomor npm satu persatu kebawah.
    \lstinputlisting[firstline=265, lastline=269]{src/1174056_praktek.py}

    \item Buatlah fungsi dengan inputan variabel NPM, didalamnya melakukan penjum-
    lahan dari seluruh dijit NPM tersebut, wajib menggunakan perulangan dan
    atau kondisi.
    \lstinputlisting[firstline=272, lastline=279]{src/1174056_praktek.py}

    \item Buatlah fungsi dengan inputan variabel NPM, didalamnya melakukan melakukan
    perkalian dari seluruh dijit NPM tersebut, wajib menggunakan perulangan dan
    atau kondisi.
    \lstinputlisting[firstline=281, lastline=288]{src/1174056_praktek.py}

    \item Buatlah fungsi dengan inputan variabel NPM, Lakukan print NPM anda tapi
    hanya dijit genap saja. wajib menggunakan perulangan dan atau kondisi.
    \lstinputlisting[firstline=290, lastline=296]{src/1174056_praktek.py}

    \item Buatlah fungsi dengan inputan variabel NPM, Lakukan print NPM anda tapi
    hanya dijit ganjil saja. wajib menggunakan perulangan dan atau kondisi.
    \lstinputlisting[firstline=298, lastline=304]{src/1174056_praktek.py}

    \item Buatlah fungsi dengan inputan variabel NPM, Lakukan print NPM anda tapi
    hanya dijit yang termasuk bilangan prima saja. wajib menggunakan perulangan
    dan atau kondisi.
    \lstinputlisting[firstline=306, lastline=320]{src/1174056_praktek.py}

    \item Buatlah satu library yang berisi fungsi-fungsi dari nomor diatas dengan nama
    le rangga.py dan berikan contoh cara pemanggilannya pada le main.py.
    \lstinputlisting[firstline=7, lastline=7]{src/mainn.py}

    \item Buatlah satu library class dengan nama le kelas3lib.py yang merupakan mod-
    ikasi dari fungsi-fungsi nomor diatas dan berikan contoh cara pemanggilannya
    pada le mainn.py.
    \lstinputlisting[firstline=8, lastline=9]{src/mainn.py}
    
\end{enumerate}
\subsubsection{Ketrampilan Penanganan Error}
Error yang di dapat dari mengerjakan tugas ini adalah type error, cara menaggulaginya dengan cara mengecheck kembali codingannya
kemudian run kembali aplikasinya
berikut contoh Penggunaan fungsi try dan exception
\lstinputlisting[firstline=177, lastline=182]{src/1174056_praktek.py}


\section{Faisal Najib Abdullah}
\subsection{Pemahanan Teori}
\begin{enumerate}
    \item Apa itu fungsi, inputan fungsi dan kembalian fungsi dengan contoh kode program
    lainnya.
    Fungsi adalah bagian dari program yang dapat digunakan ulang.
    Berikut merupakan contoh fungsi dan cara pemanggilannya
    \lstinputlisting{src/chapter3/1174042_1,1.py}

    Fungsi dapat membaca parameter, parameter adalah nilai yang disediakan kepada fungsi, dimana nilai ini akan menentukan output yang akan dihasilkan fungsi.
    \lstinputlisting{src/chapter3/1174042_1,1,1.py}

    Statemen return digunakan untuk keluar dari fungsi. Kita juga dapat menspesifikasikan nilai kembalian.
    \lstinputlisting{src/chapter3/1174042_1,1,2.py}

    \item Apa itu paket dan cara pemanggilan paket atau library dengan contoh kode
    program lainnya.
    Untuk memudahkan dalam pemanggilan fungsi yang di butuhkan, agar dapat dipanggil berulang.
    Cara pemanggilannya
    \lstinputlisting{src/chapter3/1174042_1,2.py}

    \item Jelaskan Apa itu kelas, apa itu objek, apa itu atribut, apa itu method dan
    contoh kode program lainnya masing-masing.
    kelas merupakan sebuah blueprint yang mepresentasikan objek.
    objek adalah hasil cetakan dadri sebuah kelas.
    method adalah suatu upaya yang digunakan oleh object.
    \lstinputlisting{src/chapter3/1174042_1,3.py}

    \item Jelaskan cara pemanggikan library kelas dari instansiasi dan pemakaiannya den-
    gan contoh program lainnya.
    Cara Pemanggilanya 
    \begin{itemize}
        \item pertama import terlebih dahulu filenya.
        \item kemudian buat variabel untuk menampung datanya
        \item setelah itu panggil nama classnya dan panggil methodnya
        \item Gunakan perintah print untuk menampilkan hasilnya

    \end{itemize}
    \lstinputlisting{src/chapter3/1174042_1,4.py}

    \item Jelaskan dengan contoh pemakaian paket dengan perintah from kalkulator im-
    port Penambahan disertai dengan contoh kode lainnya.
    Penggunaan paket from namafile import, itu berfungsi untuk memanggil file dan fungsinya
    \lstinputlisting{src/chapter3/1174042_1,2.py}

    \item Jelaskan dengan contoh kodenya, pemakaian paket fungsi apabila file library
    ada di dalam folder.
    Pemakaian paket adalah perkumpulan fungsi-fungsi. contoh kodenya adalah sebagai berikut :
	\lstinputlisting{src/chapter3/1174042_1,6.py}

    \item Jelaskan dengan contoh kodenya, pemakaian paket kelas apabila file library ada
    di dalam folder.
    \lstinputlisting{src/chapter3/1174042_1,6.py}
\end{enumerate}
\subsection{Ketrampilan Pemrograman}
\begin{enumerate}
    \item Buatlah fungsi dengan inputan variabel NPM, dan melakukan print luaran huruf
    yang dirangkai dari tanda bintang, pagar atau plus dari NPM kita. Tanda
    bintang untuk NPM mod 3=0, tanda pagar untuk NPM mod 3 =1, tanda plus
    untuk NPM mod3=2.
    \lstinputlisting{src/chapter3/1174042_2,1.py}

    \item Buatlah fungsi dengan inputan variabel berupa NPM. kemudian dengan meng-
    gunakan perulangan mengeluarkan print output sebanyak dua dijit belakang
    NPM.
    \lstinputlisting{src/chapter3/1174042_2,2.py}

    \item Buatlah fungsi dengan dengan input variabel string bernama NPM dan beri
    luaran output dengan perulangan berupa tiga karakter belakang dari NPM se-
    banyak penjumlahan tiga dijit tersebut.
    \lstinputlisting{src/chapter3/1174042_2,3.py}

    \item Buatlah fungsi hello word dengan input variabel string bernama NPM dan
    beri luaran output berupa digit ketiga dari belakang dari variabel NPM meng-
    gunakan akses langsung manipulasi string pada baris ketiga dari variabel NPM.
    \lstinputlisting{src/chapter3/1174042_2,4.py}

    \item buat fungsi program dengan input variabel NPM dan melakukan print nomor npm satu persatu kebawah.
    \lstinputlisting{src/chapter3/1174042_2,5.py}

    \item Buatlah fungsi dengan inputan variabel NPM, didalamnya melakukan penjum-
    lahan dari seluruh dijit NPM tersebut, wajib menggunakan perulangan dan
    atau kondisi.
    \lstinputlisting{src/chapter3/1174042_2,6.py}

    \item Buatlah fungsi dengan inputan variabel NPM, didalamnya melakukan melakukan
    perkalian dari seluruh dijit NPM tersebut, wajib menggunakan perulangan dan
    atau kondisi.
    \lstinputlisting{src/chapter3/1174042_2,7.py}

    \item Buatlah fungsi dengan inputan variabel NPM, Lakukan print NPM anda tapi
    hanya dijit genap saja. wajib menggunakan perulangan dan atau kondisi.
    \lstinputlisting{src/chapter3/1174042_2,8.py}

    \item Buatlah fungsi dengan inputan variabel NPM, Lakukan print NPM anda tapi
    hanya dijit ganjil saja. wajib menggunakan perulangan dan atau kondisi.
    \lstinputlisting{src/chapter3/1174042_2,9.py}

    \item Buatlah fungsi dengan inputan variabel NPM, Lakukan print NPM anda tapi
    hanya dijit yang termasuk bilangan prima saja. wajib menggunakan perulangan
    dan atau kondisi.
    \lstinputlisting{src/chapter3/1174042_2,10.py}

    \item Buatlah satu library yang berisi fungsi-fungsi dari nomor diatas dengan nama
    file 3lib.py dan berikan contoh cara pemanggilannya pada file main.py.
    \lstinputlisting{src/chapter3/1174042_main.py}

    \item Buatlah satu library class dengan nama file kelas3lib.py yang merupakan mod-
    ifikasi dari fungsi-fungsi nomor diatas dan berikan contoh cara pemanggilannya
    pada file main.py.
    \lstinputlisting{src/chapter3/1174042_main.py}
    
\end{enumerate}
\subsection{Ketrampilan Penanganan Error}
Error yang di dapat dari mengerjakan tugas ini adalah type error, cara menaggulaginya dengan cara mengecheck kembali codingannya
kemudian run kembali aplikasinya
berikut contoh Penggunaan fungsi try dan exception
\lstinputlisting{src/chapter3/1174042_2err.py}

\section{Yusniar Nur Syarif Sidiq/1164089}
\subsection{Teori}
\begin{enumerate}
\item Fungsi merupakan sebuah bagian dari program yang dapat digunakan ulang dan memiliki inputan variabel serta nilai yang akan di kembalikannya. Contohnya adalah source code berikut ini,
	 \lstinputlisting{src/chapter2/1164089/1164089_1.py}
Dalam dalam source code tersebut akan mengeluarkan output Hallo 1164089 ketika kita running di dalam spyder.

\item Library dalam python disini merupakan kumpulan dari fungsi dan cara pemanggilannya adalah dengan melakukan import file librarynya. Sebagai contoh, buatlah Matematika.py dan 1164089\_2.py, simpan dalam satu folder. Untuk Matematika.py isikan fungsi sebagai berikut
	 \lstinputlisting{src/chapter2/1164089/Matematika.py}
Untuk memanggil fungsi tersebut adalah dengan melakukan import  Matematika.py pada 1164089\_2.py adalah sebagai berikut,
	 \lstinputlisting{src/chapter2/1164089/1164089_2.py}

\item Class merupakan salah satu cara untuk membuat sebuah kode yang mempunyai objek serta atribut tertentu sehingga akan lebih mudah dalam mengorganisasi berbagai fungsi dan statenya. Objek disini merupakan instansiasi atau perwujudan dari sebuah class. Untuk membuat class yang memiliki objek serta atribut dapat dilihat pada source code berikut ini, dimana kita akan membaut file bernama mtk.py
	\lstinputlisting{src/chapter2/1164089/mtk.py}
Self tersebut berfungsi untuk menunjukkan variabel lokal dari class tersebut. Untuk memanggil class tersebut kita akan membuat file bernama 1164089\_3.py dan kita akan melakukan import mtk.py pada file tersebut, untuk source codenya dapat dilihat seperti berikut,
	\lstinputlisting{src/chapter2/1164089/1164089_3.py}

\item Cara memanggilnya yaitu
	\begin{itemize}
		\item Pertama import terlebih dahulu filenya
		\item Buat variabel yang berfungsi menampung data
		\item Panggil nama classnya dan methodnya
		\item Gunakan perintah print untuk menampilkannya
	\end{itemize}
Sebagai contoh perhatikan source code ini,
	\lstinputlisting{src/chapter2/1164089/1164089_4.py}

\item Dimana kita akan melakukan membuka library Matematika.py dan akan melakukan import dari fungsi di dalamnya yaitu matematika, sehingga akan lebih simple dalam penulisan source codenya adalah sebagai berikut,
	\lstinputlisting{src/chapter2/1164089/1164089_5.py}

\item

\item 
\end{enumerate}

\subsection{Keterampilan Pemrograman}
\begin{enumerate}

\item Soal No 1 \lstinputlisting{src/chapter2/1164089/1164089_21.py}

\item Soal No 2 \lstinputlisting{src/chapter2/1164089/1164089_22.py}

\item Soal No 3 \lstinputlisting{src/chapter2/1164089/1164089_23.py}

\item Soal No 4 \lstinputlisting{src/chapter2/1164089/1164089_24.py}

\item Soal No 5 \lstinputlisting{src/chapter2/1164089/1164089_25.py}

\item Soal No 6 \lstinputlisting{src/chapter2/1164089/1164089_26.py}

\item Soal No 7 \lstinputlisting{src/chapter2/1164089/1164089_27.py}

\item Soal No 8 \lstinputlisting{src/chapter2/1164089/1164089_28.py}

\item Soal No 9 \lstinputlisting{src/chapter2/1164089/1164089_29.py}

\item Soal No 10 \lstinputlisting{src/chapter2/1164089/1164089_30.py}

\item Soal No 11 \lstinputlisting{src/chapter2/1164089/1164089_31.py}

\item Soal No 12 \lstinputlisting{src/chapter2/1164089/1164089_32.py}
\end{enumerate}

\subsection{Penanganan Erorr}
\begin{enumerate}

\item Erorr yang saya temui di antaranya adalah Systax Erorr, dimana suatu keadaan script python mengalami kesalahan dalam penulisannya dan solusi dari permasalahan ini adalah dengan memperbaiki script penulisan yang salah. Untuk contoh fungsi trx except dapat dilihat pada source code berikut ini,

	\lstinputlisting{src/chapter2/1164089/1164089_33.py}

\end{enumerate}