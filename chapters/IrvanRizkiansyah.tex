\section{Python}
	\subsection{Background}
	\label{Background}
	\par
	Python adalah sebuah bahasa pemrograman yang bersifat interpreter, interactive, object-oriented, dan dapat beroperasi hampir pada semua platform seperti Windows, Linux, Mac. Python termasuk sebagai bahasa pemrograman yang dapat dengan mudah di pelajari karena sintaks yang jelas dan mudah dipahami, dan dapat dikombinasikan dengan penggunaan modul yang siap pakai, dan struktur data tingkat tinggi yang efisien \cite{prasetya2012deteksi}.
	\par
	Python memiliki kepustakaan atau biasa disebut library yang sangat luas, dan dalam distribusi Python yang telah disediakan, hal tersebut diakibatkan oleh pendistribusian Python yang bebas karena bahasa pemrograman Python merupakan bahasa pemrograman yang freeware atau bebas dalam hal pengembangannya. Python adalah sebuah bahasa pemrograman yang dapat dengan mudah dibaca dan terstruktur, hal tersebut dikarenakan penggunaan sistem identasi, yaitu pemisahan blok-blok program susunan identasi, jadi untuk menambahkan sub-sub program dalam sebuah blok program, sub program tersebut harus diletakkan pada satu atau lebih spasi dari kolom sebuah blok \cite{perkasa2014rancang}.
	\par
	Bahasa pemrograman Python dibuat oleh Guido Van Rossum. Dikarenakan para pengembang software atau perangkat lunak lebih cenderung memilih kecepatan dalam menyelesaikan suatu proyek dibandingkan dengan kecepatam proses dari program yang dijalankan, maka dari itu bahasa pemrograman Python dapat dibilang bahasa pemerograman yang kecepatannya dapat melebihi bahasa pemrograman C. Akan tetapi bahasa pemrograman Python lebih lambat dalam memproses suatu program dibandingkan bahasa pemrograman C. dengan berkembangnya kecepatan prosesor dan memori saat ini, mengakibatkan tidak terlihatnya keterlambatan dari sebuah program yang menggunakan bahasa pemrograman Python \cite{miftakhuddinimplementasi}.

	\subsection{Problems}
		\begin{itemize}
			\item Kurangnya pemahaman tentang bahasa pemrograman Python
			\item Kurang mengerti dalam hal fungsi-fungsi yang terdapat pada bahasa pemrograman Python
		\end{itemize}

	\subsection{Objective and Contribution}
		\subsubsection{Objective}
			\begin{itemize}
				\item Dapat memahami tentang bahasa pemrograman Python
				\item Dapat memahami fungsi fungsi yang terdapat pada bahasa pemrograman Python
			\end{itemize}
	
		\subsubsection{Contribution}
			\begin{itemize}
				\item Dapat membangun sebuah sistem dengan menggunakan bahasa pemrograman Python
				\item Dapat membangun sebuah alat yang berguna, menggunakan mikrokontroler dan bahasa pemrograman python
			\end{itemize}

	\subsection{Scoop and Environtment}
		\begin{itemize}
			\item Pengenanalan tentang bahasa pemrograman Python
			\item Pengenalan fungsi-fungsi yang terdapat pada bahasa pemrograman Python
		\end{itemize}
