\section{Python}
\subsection{Background}
\label{Background}
\par
Python adalah bahasa pemrograman interpretatif multiguna dengan filosofi desain yang berfokus pada keterbacaan kode dan python sendiri diklaim sebagai bahasa yang menggabungkan kapabilitas, kemampuan, dengan kode sintaksis yang sangat jelas, dan dilengkapi dengan fungsi pustaka standar yang besar dan komprehensif. Python juga didukung oleh komunitas besar.
\par
Python mendukung pemrograman multi-paradigma, terutama tetapi tidak terbatas pada pemrograman berorientasi objek, pemrograman imperatif, dan pemrograman fungsional. Salah satu fitur yang tersedia di Python adalah sebagai bahasa pemrograman dinamis yang dilengkapi dengan manajemen memori otomatis. Python menggunakan bahasa bahasa scripting yang sama seperti bahasa pemrograman dinamis, meskipun dalam praktiknya penggunaan bahasa ini lebih luas mencakup konteks penggunaan yang umumnya tidak dilakukan menggunakan bahasa skrip. Python dapat digunakan untuk keperluan pengembangan perangkat lunak dan dapat berjalan di berbagai platform sistem operasi.
\par
CPython, implementasi referensi Python, adalah perangkat lunak bebas dan open source dan memiliki model pengembangan berbasis komunitas, seperti halnya hampir semua implementasi alternatifnya. CPython dikelola oleh Yayasan Perangkat Lunak Python nirlaba \cite{monk2013programming}.

\subsection{Problems}
\begin{itemize}
	\item Mahasiswa D4 TI belum dapat belum memahami apa itu python
    \item Mahasiswa D4 TI belum mengerti fungsi fungsi apa saja yang terdapat pada python
    \item Mahasiswa D4 TI belum dapat menjalankan fungsi python
\end{itemize}

\subsection{Objective and Contribution}
\subsubsection{Objective}
\begin{itemize}
	\item Mahasiswa D4 TI dapat memahami apa itu python
	\item Mahasiswa D4 TI dapat memahami fungsi fungsi yang terdapat pada python
	\item Mahasiswa D4 TI dapat menjalankan fungsi python
\end{itemize}
	
\subsubsection{Contribution}
\begin{itemize}
	\item Mahasiswa D4 TI dapat membangun suatu aplikasi yang mengimplementasikan bahasa python
	\item Mahasiswa D4 TI dapat membangun alat yang terhubung dengan aplikasi menggunakan bahasa python
\end{itemize}

\subsection{Scoop and Environtment}
\begin{itemize}
	\item Mengenali apa itu python pada mahasiswa
	\item Mengenali fungsi fungsi dasar python dan menjalankannya
\end{itemize}
