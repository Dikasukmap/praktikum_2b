\section{Luthfi Muhammad Nabil\_1174035}
\subsection{Latar Belakang}
Python adalah sebuah bahasa pemrograman dengan level tinggi yang interaktif, dan mendukung berbagai paradigma pemrograman. Python sudah terkenal pada kalangan 
programmer sebagai bahasa yang mudah dipahami dan sudah banyak komunitas pendukung karena penggunanya yang banyak. Selain pada komunitas biasa, Python 
sudah diimplementasikan pada banyak perusahaan ternama dan dipasang pada aplikasi yang sudah terkenal seperti search engine google pada perusahaan Google. 
Python mulai dirilis pada tahun 1991 oleh Guido van Rossum sebagai kelanjutan dari bahasa pemrograman ABC dengan memiliki versi yaitu 0.9.0. Nama dari bahasa Python diambil dari program televisi di Inggris bernama Monty Python. Lalu tahun 1995, Guido pindah ke CNRI di Virginia, Amerika sembari melanjutkan pengembangan Python. Versi terakhir yang dikeluarkan telah mencapai 1.6. 
Lalu pada tahun 2000, dirilis Python versi 2.0 yang memiliki peran sebagai bahasa pemrograman tidak berbayar atau open source. Van Rossum sendiri aktif pada development dari Python tetapi sudah bergabung dengan banyak penyumbang. Dibandingkan dengan bahasa lain, Python sudah melewati beberapa versi yang terbatas, mengikuti filosofi dari perubahan berurutan. 