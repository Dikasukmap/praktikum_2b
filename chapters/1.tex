\section{Irvan Rizkiansyah}

\section{Python}
	\subsection{Background}
	\label{Background}
	\par
	Python adalah sebuah bahasa pemrograman yang bersifat interpreter, interactive, object-oriented, dan dapat beroperasi hampir pada semua platform seperti Windows, Linux, Mac. Python termasuk sebagai bahasa pemrograman yang dapat dengan mudah di pelajari karena sintaks yang jelas dan mudah dipahami, dan dapat dikombinasikan dengan penggunaan modul yang siap pakai, dan struktur data tingkat tinggi yang efisien \cite{prasetya2012deteksi}.
	\par
	Python memiliki kepustakaan atau biasa disebut library yang sangat luas, dan dalam distribusi Python yang telah disediakan, hal tersebut diakibatkan oleh pendistribusian Python yang bebas karena bahasa pemrograman Python merupakan bahasa pemrograman yang freeware atau bebas dalam hal pengembangannya. Python adalah sebuah bahasa pemrograman yang dapat dengan mudah dibaca dan terstruktur, hal tersebut dikarenakan penggunaan sistem identasi, yaitu pemisahan blok-blok program susunan identasi, jadi untuk menambahkan sub-sub program dalam sebuah blok program, sub program tersebut harus diletakkan pada satu atau lebih spasi dari kolom sebuah blok \cite{perkasa2014rancang}.
	\par
	Bahasa pemrograman Python dibuat oleh Guido Van Rossum. Dikarenakan para pengembang software atau perangkat lunak lebih cenderung memilih kecepatan dalam menyelesaikan suatu proyek dibandingkan dengan kecepatam proses dari program yang dijalankan, maka dari itu bahasa pemrograman Python dapat dibilang bahasa pemerograman yang kecepatannya dapat melebihi bahasa pemrograman C. Akan tetapi bahasa pemrograman Python lebih lambat dalam memproses suatu program dibandingkan bahasa pemrograman C. dengan berkembangnya kecepatan prosesor dan memori saat ini, mengakibatkan tidak terlihatnya keterlambatan dari sebuah program yang menggunakan bahasa pemrograman Python \cite{miftakhuddinimplementasi}.

	\subsection{Problems}
		\begin{itemize}
			\item Kurangnya pemahaman tentang bahasa pemrograman Python
			\item Kurang mengerti dalam hal fungsi-fungsi yang terdapat pada bahasa pemrograman Python
		\end{itemize}

	\subsection{Objective and Contribution}
		\subsubsection{Objective}
			\begin{itemize}
				\item Dapat memahami tentang bahasa pemrograman Python
				\item Dapat memahami fungsi fungsi yang terdapat pada bahasa pemrograman Python
			\end{itemize}
	
		\subsubsection{Contribution}
			\begin{itemize}
				\item Dapat membangun sebuah sistem dengan menggunakan bahasa pemrograman Python
				\item Dapat membangun sebuah alat yang berguna, menggunakan mikrokontroler dan bahasa pemrograman python
			\end{itemize}

	\subsection{Scoop and Environtment}
		\begin{itemize}
			\item Pengenanalan tentang bahasa pemrograman Python
			\item Pengenalan fungsi-fungsi yang terdapat pada bahasa pemrograman Python
		\end{itemize}

\section{Luthfi Muhammad Nabil\_1174035}
\subsection{Background}
Python adalah sebuah bahasa pemrograman dengan level tinggi yang interaktif, dan mendukung berbagai paradigma pemrograman. Python sudah terkenal pada kalangan programmer sebagai bahasa yang mudah dipahami dan memiliki kompleksitas yang dinamis sehingga dapat dipakai di algoritma maupun platform yang berbagai macam.Python sudah memiliki banyak komunitas pendukung karena penggunanya yang banyak. Selain pada komunitas biasa, Python sudah diimplementasikan pada banyak perusahaan ternama dan dipasang pada aplikasi yang sudah terkenal seperti pada search engine google yang dimiliki oleh perusahaan Google. 
\linebreak
\linebreak
Python mulai dirilis pada tahun 1991 oleh Guido van Rossum sebagai kelanjutan dari bahasa pemrograman ABC dengan memiliki versi yaitu 0.9.0. Nama dari bahasa Python diambil dari program televisi di Inggris bernama Monty Python. Lalu tahun 1995, Guido pindah ke CNRI di Virginia, Amerika sembari melanjutkan pengembangan Python. Versi terakhir yang dikeluarkan telah mencapai 1.6. Pada awalnya, Python adalah bahasa yang dipakai untuk  Lalu pada tahun 2000, dirilis Python versi 2.0 yang memiliki peran sebagai bahasa pemrograman tidak berbayar atau open source. Van Rossum sendiri aktif pada development dari Python tetapi sudah bergabung dengan banyak penyumbang. Dibandingkan dengan bahasa lain, Python sudah melewati beberapa versi yang terbatas, mengikuti filosofi dari perubahan berurutan. 
\linebreak
\linebreak
Untuk memahami bahasa Python tidak sulit, tetapi instalasi Python cukup memiliki trik tersendiri terlebih untuk pengguna yang baru memasuki lingkup programming. Pada sistem operasi windows, pengguna diharuskan untuk memasuki sistem pada windows untuk mengatur lokasi dari Python yang sudah diinstall. Selain itu, untuk yang terbiasa dengan beberapa pemrograman harus beradaptasi dengan aturan - aturan pada bahasa pemrograman Python seperti penggantian titik koma (;) dengan indentasi. Oleh karena itu, penulis akan membahas mengenai pengenalan singkat mengenai bahasa pemrograman python dan cara instalasi dari python dan library pip.

\subsection{Problems}
Sesuai dengan latar belakang yang telah dibahas, penulis merumuskan masalah sebagai berikut : 
\begin{enumerate}	
	\item Bagaimana pemaparan singkat mengenai Python?
	\item Bagaimana cara melakukan instalasi Python?
\end{enumerate}

\subsection{Objective and Contribution}
\subsubsection{Objective}
\begin{enumerate}
	\item Untuk membahas mengenai Python.
	\item Untuk menunjukkan cara instalasi Python.
\end{enumerate}

\subsubsection{Contribution}
Pada materi ini, penulis menggunakan Python.

\subsection{Scoop and Environment}
\begin{itemize}
	\item Pada Chapter 1 membahas mengenai sejarah, latar belakang, dan keterangan singkat mengenai python tersebut. Chapter ini juga merangkum masalah dan mencari tujuan yang ingin dicapai penulis dalam membuat resume ini.
\end{itemize}

\section{Hagan Rowlenstino/1174040}
\subsection{Background}
Python di desain sebagai bahasa pemrograman yang dapat digunakan sehari-hari. Pencipta python ,Guido van Rossum, telah menulis seri lengkap tentang sejarah bahasa tersebut.Python diciptakan di awal 1990 di CWI \'(the Centrum voor Wiskunde and Informatica), tempat kelahiran ALGOL \'(Algorithmic Language 68 ). Sebelumnya, Rossum juga telah mengerjakan bahasa pemrograman ABC, yang dikembangkan di  CWI sebagai bahasa pengajaran yang menekankan kejelasan. Walaupun project ABC telah di tutup , Rossum banyak belajar dari hal tersebut saat dia mulai membuat Python sebagai alat untuk multimedia dan project penelitian sistem operasi. Dia ingin Python mempunyai tingkatan yang cukup tinggi agar mudah untuk dibaca dan ditulis, juga mirip dengan Java, dan menawarkan portabilitas serta error model yang terdefinisi dengan baik.
\linebreak
\linebreak
Python juga kaya akan vocabulary yang berguna untuk membuat algoritma yang kompleks dengan efisien dikarenakan punya dictionaries yang memiliki string yang kuat dan assosiasi array yang fleksibel. Python menggabungkan antara fleksibilitas tingkat tinggi, kemampuan membaca, dan interface yang terdefinisi dengan baik. Kombinasi tersebut membuat Python cocok untuk menyelesaikan masalah komputasi non-algoritma seperti integrase dengan web, format data, ataw hardware kelas rendah. Python mudah untuk dipelajari karena strukturnya sederhana dan sintaksnya jelas, punya library yang portable dan dapat digunakan di beda perangkat,dan dapat terintegrasi dengan bahasa pemrograman lain seperti C, C++, dan Java.
\subsection{Problems}
\begin{enumerate}
\item Banyak pemrograman yang penggunaannya kompleks
\end{enumerate}
\subsection{Objective and Contribution}
\subsubsection{Objective}
\begin{enumerate}
\item Dapat memudahkan pemrograman dengan bahasa pemrograman yang tepat
\end{enumerate}
\subsubsection{Contribution}
\begin{enumerate}
\item Menggunakan Python sebagai bahasa pemrograman
\end{enumerate}
\subsection{Scoop and Environment}
\begin{enumerate}
\item Mengimplementasikan Python dalam pemrograman
\end{enumerate}

\section{Rangga Putra Ramdhani\_1174056}
\subsection{Background}
	\par
	Python adalah bahasa pemrograman tingkat tinggi untuk keperluan umum yang filosofi desainnya menekankan keterbacaan kode. Sintaksis Python memungkinkan programmer untuk mengekspresikan konsep dalam lebih sedikit baris kode daripada yang mungkin dilakukan dalam bahasa seperti C dan bahasa tersebut menyediakan konstruksi yang dimaksudkan untuk memungkinkan program yang jelas pada skala kecil dan besar.
	\par
	Python mendukung banyak paradigma pemrograman, termasuk gaya pemrograman berorientasi objek, imperatif dan fungsional. Ini fitur sistem tipe yang sepenuhnya dinamis dan manajemen memori otomatis, mirip dengan Skema, Ruby, Perl dan Tclm dan memiliki perpustakaan standar yang besar dan komprehensif.
	\par
	Seperti bahasa dinamis lainnya, Python sering digunakan sebagai bahasa scripting, tetapi juga digunakan dalam berbagai konteks non-scripting. Menggunakan alat pihak ketiga, kode Python dapat dikemas ke dalam program yang dapat dieksekusi mandiri. Penerjemah python tersedia untuk banyak sistem operasi.

\subsection{Problems}
	\begin{enumerate}
		\item Bagaimana mahasiswa politeknik pos indonesia dapat menggunakan bahasa python.
		\item Kenapa mahasiswa politeknik pos indonesia harus belajar bahasa pemrograman python.
		\item Bagaimana cara menggunakan bahasa python terhadap web service.
	\end{enumerate}

\subsection{Objective and Contribution}
	\subsubsection{Objective}
		\begin{enumerate}
			\item Mahasiswa politeknik pos indonesia dapat memahami bahasa python secara bertahap.
			\item Bahasa pemrograman python dapat dijalankan di Linux, Mac dan Windows.
			\item Menggunakan bahasa python dapat mempermudah mahasiswa dalam membuat web service.
		\end{enumerate}
	\subsubsection{Contribution}
		\begin{enumerate}
			\item Membantu mahasiswa politeknik pos indonesia dalam menyelesaikan masalah pada python.
			\item Membantu mahasiswa politeknik pos indonesia memahami bahasa pemrograman python.
			\item Mempelajari bahasa python dengan proses pembuatan web service.
		\end{enumerate}
			
\subsection{Scope and Environtment}
		\begin{enumerate}
			\item Mahasiswa politeknik pos indonesia dapat memahami bahasa python.
			\item Mahasiswa politeknik pos indonesia dapat menjalankan fungsi python.
			\item Mahasiswa politeknik pos indonesia dapat membuat web service dengan menggunakan python.
		\end{enumerate}

