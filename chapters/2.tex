
\section{Rangga Putra Ramdhani}
\subsection{Teori}
\begin{enumerate}
    \item Jenis jenis variable phyton dan cara pemakaiannya
Variabel adalah tempat menyimpan nya objek informasi informasi yang dinamis. Dalam Phyton terdapat beberapa variabel dengan berbagai type data diantaranya adalah variabel dengan type data number, string, dan boolean. Dalam phyton kita dapat membuat variable dengan cara sebagai gambar berikut
   \lstinputlisting[firstline=8, lastline=12]{src/1174056_teori.py}
    \item Kode untuk meminta input dari user dan bagaimana melakukan output ke layar
 \lstinputlisting[firstline=67, lastline=68]{src/1174056_teori.py}
    \item Operator dasar aritmatika
Ada operator penambahan, pengurangan perkalian, perkalian, pembagian, modulus, perpangkatan, dan pembulatan decimal.
\lstinputlisting[firstline=71, lastline=94]{src/1174056_teori.py}
    \item Perulangan
Ada dua jenis perulangan di dalam phyton yaitu perulangan while dan perulangan for
 \lstinputlisting[firstline=97, lastline=99]{src/1174056_teori.py}
 \lstinputlisting[firstline=102, lastline=105]{src/1174056_teori.py}
    \item sintak Untuk memilih kondisi, dan kondisi didalam kondisi
Pengambilan kondisi If yang digunakan untuk mengantisipasi kondisi yang terjadi saat program dijalankan dan menentukan tindakan apa yang akan diambil sesuai dengan kondisi.
  \lstinputlisting[firstline=108, lastline=111]{src/1174056_teori.py}
  \lstinputlisting[firstline=114, lastline=119]{src/1174056_teori.py}
  \lstinputlisting[firstline=122, lastline=129]{src/1174056_teori.py}

    \item Jenis-jenis error pada phyton
Syntax Errors adalah keadaan dimana kode python mengalami kesalahan dalam penulisan. 
ZeroDivisonError adalah eror yang terjadi saat eksekusi program menghasilkan perhitungan matematika pembagian dengan angka nol.
NameError adalah eror yang terjadi saat kode di eksekusi terhadap local name atau global name yang tidak terdefinisi. 
TypeError adalah eror yang terjadi saat dilakukan eksekusi pada suatu operasi atau fungsi dengan type object yang tidak sesuai.

    \item Cara memakai try except
Cara pemakaian try except adalah sebagai berikut :

\lstinputlisting[firstline=132, lastline=138]{src/1174056_teori.py}

\end{enumerate}

\subsection{praktek}
\begin{enumerate}
    \item Jawaban soal no 1
    \lstinputlisting[firstline=11, lastline=20]{src/1174056_praktek.py}
    \item Jawaban soal no 2
    \lstinputlisting[firstline=24, lastline=28]{src/1174056_praktek.py}
    \item Jawaban soal no 3
    \lstinputlisting[firstline=33, lastline=37]{src/1174056_praktek.py}
    \item Jawaban soal no 4
    \lstinputlisting[firstline=40, lastline=41]{src/1174056_praktek.py}
    \item Jawaban soal no 5
    \lstinputlisting[firstline=44, lastline=56]{src/1174056_praktek.py}
    \item Jawaban soal no 6
    \lstinputlisting[firstline=59, lastline=60]{src/1174056_praktek.py}
    \item Jawaban soal no 7
    \lstinputlisting[firstline=63, lastline=64]{src/1174056_praktek.py}
    \item Jawaban soal no 8
    \lstinputlisting[firstline=67, lastline=71]{src/1174056_praktek.py}
    \item Jawaban soal no 9
    \lstinputlisting[firstline=74, lastline=74]{src/1174056_praktek.py}
    \item Jawaban soal no 10
    \lstinputlisting[firstline=77, lastline=77]{src/1174056_praktek.py}
    \item Jawaban soal no 11
    \lstinputlisting[firstline=80, lastline=80]{src/1174056_praktek.py}
\end{enumerate}

\subsection{Keterampilan dan penanganan eror}
    \lstinputlisting[firstline=10, lastline=17]{src/ror.py}

\section{Dika Sukma Pradana 1174050}
\subsection{Teori}
\begin{enumerate}
	\item Jenis-jenis variable phyton dan cara pemakaiannya
Variabel merupakan tempat menyimpan data. Dalam Phyton terdapat beberapa variabel dengan berbagai type data diantaranya adalah variabel dengan type data number, string, dan boolean. Dalam phyton kita dapat membuat variable dengan cara sebagai gambar berikut
   \lstinputlisting[firstline=8, lastline=12]{src/1174050_teori.py}
	\item Kode input user dan melakukan output ke layar
 \lstinputlisting[firstline=67, lastline=68]{src/1174050_teori.py}
	\item Operator dasar aritmatika
Macam operator penambahan, pengurangan perkalian, perkalian, pembagian, modulus, perpangkatan, dan pembulatan decimal.
\lstinputlisting[firstline=71, lastline=94]{src/1174050_teori.py}
	\item Perulangan
Macam perulangan di dalam phyton yaitu perulangan while dan perulangan for
 \lstinputlisting[firstline=97, lastline=99]{src/1174050_teori.py}
 \lstinputlisting[firstline=102, lastline=105]{src/1174050_teori.py}
	\item sintak Untuk memilih kondisi, dan kondisi didalam kondisi
Pengambilan kondisi If yang digunakan untuk mengantisipasi kondisi yang terjadi saat program dijalankan dan menentukan tindakan apa yang akan diambil sesuai dengan kondisi.
  \lstinputlisting[firstline=108, lastline=111]{src/1174050_teori.py}
  \lstinputlisting[firstline=114, lastline=119]{src/1174050_teori.py}
  \lstinputlisting[firstline=122, lastline=129]{src/1174050_teori.py}

	\item Jenis-jenis error pada phyton
Syntax Errors adalah keadaan dimana kode python mengalami kesalahan penulisan. 
ZeroDivisonError adalah eror yang terjadi saat eksekusi program menghasilkan perhitungan matematika pembagian dengan angka nol.
NameError adalah eror yang terjadi saat kode di eksekusi terhadap local name atau global name yang tidak terdefinisi. 
TypeError adalah eror yang terjadi saat dilakukan eksekusi pada suatu operasi atau fungsi dengan type object yang tidak sesuai.

	\item Cara memakai try except
Cara pemakaian try except :
\lstinputlisting[firstline=132, lastline=138]{src/1174050_teori.py}

\end{enumerate}

\subsection{Praktek}
\begin{enumerate}
	\item Jawaban soal no 1
	\lstinputlisting[firstline=11, lastline=20]{src/1174050_praktek.py}
	\item Jawaban soal no 2
	\lstinputlisting[firstline=24, lastline=28]{src/1174050_praktek.py}
	\item Jawaban soal no 3
	\lstinputlisting[firstline=33, lastline=37]{src/1174050_praktek.py}
	\item Jawaban soal no 4
	\lstinputlisting[firstline=40, lastline=41]{src/1174050_praktek.py}
	\item Jawaban soal no 5
	\lstinputlisting[firstline=44, lastline=56]{src/1174050_praktek.py}
	\item Jawaban soal no 6
	\lstinputlisting[firstline=59, lastline=60]{src/1174050_praktek.py}
	\item Jawaban soal no 7
	\lstinputlisting[firstline=63, lastline=64]{src/1174050_praktek.py}
	\item Jawaban soal no 8
	\lstinputlisting[firstline=67, lastline=71]{src/1174050_praktek.py}
	\item Jawaban soal no 9
	\lstinputlisting[firstline=74, lastline=74]{src/1174050_praktek.py}
	\item Jawaban soal no 10
	\lstinputlisting[firstline=77, lastline=77]{src/1174050_praktek.py}
	\item Jawaban soal no 11
	\lstinputlisting[firstline=80, lastline=80]{src/1174050_praktek.py}
\end{enumerate}

\subsection{Keterampilan dan Penanganan Eror}
	\lstinputlisting[firstline=10, lastline=17]{src/errord1ka.py}
	
	
\section{Ichsan Hizman Hardy}
\subsection{Teori}
\begin{enumerate}
    \item Jenis jenis Variable phyton dan cara pemakaiannya 
Variabel merupakan tempat menyimpan data, Dalam Phyton terdapat beberapa variabel dengan berbagai type data diantaranya adalah variabel dengan type data number, string, dan boolean. Dalam phyton kita dapat membuat variable dengan cara sebagai gambar berikut
   \lstinputlisting[firstline=8, lastline=12]{src/1174034_ichsan_teori.py}
    \item Kode untuk meminta input dari user dan bagaimana melakukan output ke layar
 \lstinputlisting[firstline=67, lastline=68]{src/1174034_ichsan_teori.py}
    \item Operator dasar aritmatika
Ada operator penambahan, pengurangan perkalian, perkalian, pembagian, modulus, perpangkatan, dan pembulatan decimal.
\lstinputlisting[firstline=71, lastline=94]{src/1174034_ichsan_teori.py}
    \item Perulangan
Terdapat dua jenis perulangan di dalam phyton yaitu perulangan while dan perulangan for
 \lstinputlisting[firstline=97, lastline=99]{src/1174034_ichsan_teori.py}
 \lstinputlisting[firstline=102, lastline=105]{src/1174034_ichsan_teori.py}
    \item sintak Untuk memilih kondisi, dan kondisi didalam kondisi
Pengambilan kondisi If yang digunakan untuk mengantisipasi kondisi yang terjadi saat program dijalankan dan menentukan tindakan apa yang akan diambil sesuai dengan kondisi.
  \lstinputlisting[firstline=108, lastline=111]{src/1174034_ichsan_teori.py}
  \lstinputlisting[firstline=114, lastline=119]{src/1174034_ichsan_teori.py}
  \lstinputlisting[firstline=122, lastline=129]{src/1174034_ichsan_teori.py}

    \item Jenis-jenis error pada phyton
Syntax Errors adalah keadaan dimana kode python mengalami kesalahan penulisan. 
ZeroDivisonError yaitu eror yang terjadi saat eksekusi program menghasilkan perhitungan matematika pembagian dengan angka nol.
NameError yaitu eror yang terjadi saat kode di eksekusi terhadap local name atau global name yang tidak terdefinisi. 
TypeError yaitu eror yang terjadi saat dilakukan eksekusi pada suatu operasi atau fungsi dengan type object yang tidak sesuai.

    \item Cara memakai try except
Cara pemakaian try except adalah sebagai berikut :
\lstinputlisting[firstline=132, lastline=138]{src/1174034_ichsan_teori.py}

\end{enumerate}

\subsection{praktek}
\begin{enumerate}
    \item Jawaban soal no 1
    \lstinputlisting[firstline=11, lastline=20]{src/1174034_ichsan_praktek.py}
    \item Jawaban soal no 2
    \lstinputlisting[firstline=24, lastline=28]{src/1174034_ichsan_praktek.py}
    \item Jawaban soal no 3
    \lstinputlisting[firstline=33, lastline=37]{src/1174034_ichsan_praktek.py}
    \item Jawaban soal no 4
    \lstinputlisting[firstline=40, lastline=41]{src/1174034_ichsan_praktek.py}
    \item Jawaban soal no 5
    \lstinputlisting[firstline=44, lastline=56]{src/1174034_ichsan_praktek.py}
    \item Jawaban soal no 6
    \lstinputlisting[firstline=59, lastline=60]{src/1174034_ichsan_praktek.py}
    \item Jawaban soal no 7
    \lstinputlisting[firstline=63, lastline=64]{src/1174034_ichsan_praktek.py}
    \item Jawaban soal no 8
    \lstinputlisting[firstline=67, lastline=71]{src/1174034_ichsan_praktek.py}
    \item Jawaban soal no 9
    \lstinputlisting[firstline=74, lastline=74]{src/1174034_ichsan_praktek.py}
    \item Jawaban soal no 10
    \lstinputlisting[firstline=77, lastline=77]{src/1174034_ichsan_praktek.py}
    \item Jawaban soal no 11
    \lstinputlisting[firstline=80, lastline=80]{src/1174034_ichsan_praktek.py}
\end{enumerate}

\subsection{Keterampilan dan penanganan eror}
    \lstinputlisting[firstline=10, lastline=17]{src/errorrr.py}
