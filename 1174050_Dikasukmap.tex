\section{INTRODUCTION}
\subsection{Background}
	\par 
	Python adalah bahasa pemrograman tingkat tinggi untuk keperluan umum yang filosofi desainnya menekankan keterbacaan kode. Sintaksis Python memungkinkan programmer untuk mengekspresikan konsep dalam lebih sedikit baris kode daripada yang mungkin dilakukan dalam bahasa seperti C, dan bahasa tersebut menyediakan konstruksi yang dimaksudkan untuk memungkinkan program yang jelas pada skala kecil dan besar.
	\par
	Python mendukung banyak paradigma pemrograman, termasuk gaya pemrograman berorientasi objek, imperatif dan fungsional. Ini fitur sistem tipe yang sepenuhnya dinamis dan manajemen memori otomatis, mirip dengan Skema, Ruby, Perl dan Tclm dan memiliki perpustakaan standar yang besar dan komprehensif.
	\par
	Seperti bahasa dinamis lainnya, Python sering digunakan sebagai bahasa scripting, tetapi juga digunakan dalam berbagai konteks non-scripting. Menggunakan alat pihak ketiga, kode Python dapat dikemas ke dalam program yang dapat dieksekusi mandiri. Penerjemah python tersedia untuk banyak sistem operasi.
	\par
	CPython, implementasi referensi Python, adalah perangkat lunak bebas dan open source dan memiliki model pengembangan berbasis komunitas, seperti halnya hampir semua implementasi alternatifnya. CPython dikelola oleh Yayasan Perangkat Lunak Python nirlaba\cite{van2007python}.
	
\subsection{Problems}
	\begin{enumerate}
		\item Bagaimana mahasiswa D4TI2B bisa menggunakan bahasa python.
		\item Bagaimana pengaruh bahasa python terhadap mahasiswa D4TI2B.
		\item Bagaimana penggunaan bahasa python terhadap web service.
	\end{enumerate}

\subsection{Objective and Contribution}
	\subsubsection{Objective}
		\begin{enumerate}
			\item Mahasiswa D4TI2B mampu memahami bahasa pemrograman python secara bertahap.
			\item Bahasa pemrograman python mampu mempengaruhi mahasiswa D4TI2B menjadi lebih semangat dalam belajar web service.
			\item Penggunaan bahasa python mampu mempermudah mahasiswa dalam membuat web service.
		\end{enumerate}
	\subsubsection{Contribution}
		\begin{enumerate}
			\item Membantu mahasiswa D4TI2B dalam menyelesaikan masalah pada python.
			\item Membantu mahasiswa D4TI2B memahami bahasa pemrograman python.
			\item Mempelajari bahasa python dalam proses pembuatan web service.
		\end{enumerate}
\subsection{Scope and Environtment}
	\begin{enumerate}
		\item Mahasiswa D4TI2B memahami bahasa pemrograman python.
		\item Mahasiswa D4TI2B mampu menjalankan fungsi python.
		\item Mahasiswa D4TI2B mampu membuat web service menggunakan python.
	\end{enumerate}

