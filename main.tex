%%%%%%%%%%%%%%
%% Run LaTeX on this file several times to get Table of Contents,
%% cross-references, and citations.

%% If you have font problems, you may edit the w-bookps.sty file
%% to customize the font names to match those on your system.

%% w-bksamp.tex. Current Version: Feb 16, 2012
%%%%%%%%%%%%%%%%%%%%%%%%%%%%%%%%%%%%%%%%%%%%%%%%%%%%%%%%%%%%%%%%
%
%  Sample file for
%  Wiley Book Style, Design No.: SD 001B, 7x10
%  Wiley Book Style, Design No.: SD 004B, 6x9
%
%
%  Prepared by Amy Hendrickson, TeXnology Inc.
%  http://www.texnology.com
%%%%%%%%%%%%%%%%%%%%%%%%%%%%%%%%%%%%%%%%%%%%%%%%%%%%%%%%%%%%%%%%

%%%%%%%%%%%%%
% 7x10
%\documentclass{wileySev}

% 6x9
\documentclass{wileySix}

\usepackage{graphicx}
\usepackage{listings}

\usepackage{color}
 
\definecolor{codegreen}{rgb}{0,0.6,0}
\definecolor{codegray}{rgb}{0.5,0.5,0.5}
\definecolor{codepurple}{rgb}{0.58,0,0.82}
\definecolor{backcolour}{rgb}{0.95,0.95,0.92}
 
\lstdefinestyle{mystyle}{
    backgroundcolor=\color{backcolour},   
    commentstyle=\color{codegreen},
    keywordstyle=\color{magenta},
    numberstyle=\tiny\color{codegray},
    stringstyle=\color{codepurple},
    basicstyle=\footnotesize,
    breakatwhitespace=false,         
    breaklines=true,                 
    captionpos=b,                    
    keepspaces=true,                 
    numbers=left,                    
    numbersep=5pt,                  
    showspaces=false,                
    showstringspaces=false,
    showtabs=false,                  
    tabsize=2,
    language=sh
}
 
\lstset{style=mystyle}

%%%%%%%
%% for times math: However, this package disables bold math (!)
%% \mathbf{x} will still work, but you will not have bold math
%% in section heads or chapter titles. If you don't use math
%% in those environments, mathptmx might be a good choice.

% \usepackage{mathptmx}

% For PostScript text
\usepackage{w-bookps}

%%%%%%%%%%%%%%%%%%%%%%%%%%%%%%%%%%%%%%%%%%%%%%%%%%%%%%%%%%%%%%%%
%% Other packages you might want to use:

% for chapter bibliography made with BibTeX
% \usepackage{chapterbib}

% for multiple indices
% \usepackage{multind}

% for answers to problems
% \usepackage{answers}

%%%%%%%%%%%%%%%%%%%%%%%%%%%%%%
%% Change options here if you want:
%%
%% How many levels of section head would you like numbered?
%% 0= no section numbers, 1= section, 2= subsection, 3= subsubsection
%%==>>
\setcounter{secnumdepth}{3}

%% How many levels of section head would you like to appear in the
%% Table of Contents?
%% 0= chapter titles, 1= section titles, 2= subsection titles, 
%% 3= subsubsection titles.
%%==>>
\setcounter{tocdepth}{2}

%% Cropmarks? good for final page makeup
%% \docropmarks

%%%%%%%%%%%%%%%%%%%%%%%%%%%%%%
%
% DRAFT
%
% Uncomment to get double spacing between lines, current date and time
% printed at bottom of page.
% \draft
% (If you want to keep tables from becoming double spaced also uncomment
% this):
% \renewcommand{\arraystretch}{0.6}
%%%%%%%%%%%%%%%%%%%%%%%%%%%%%%

%%%%%%% Demo of section head containing sample macro:
%% To get a macro to expand correctly in a section head, with upper and
%% lower case math, put the definition and set the box 
%% before \begin{document}, so that when it appears in the 
%% table of contents it will also work:

\newcommand{\VT}[1]{\ensuremath{{V_{T#1}}}}

%% use a box to expand the macro before we put it into the section head:

\newbox\sectsavebox
\setbox\sectsavebox=\hbox{\boldmath\VT{xyz}}

%%%%%%%%%%%%%%%%% End Demo


\begin{document}


\booktitle{Cerdas Menguasai Python}
\subtitle{Dalam 24 Jam}

\authors{Rolly M. Awangga\\
\affil{Informatics Research Center}
%Floyd J. Fowler, Jr.\\
%\affil{University of New Mexico}
}

\offprintinfo{Cerdas Menguasai Python, First Edition}{Rolly M. Awangga}

%% Can use \\ if title, and edition are too wide, ie,
%% \offprintinfo{Survey Methodology,\\ Second Edition}{Robert M. Groves}

%%%%%%%%%%%%%%%%%%%%%%%%%%%%%%
%% 
\halftitlepage

\titlepage


\begin{copyrightpage}{2019}
%Survey Methodology / Robert M. Groves . . . [et al.].
%\       p. cm.---(Wiley series in survey methodology)
%\    ``Wiley-Interscience."
%\    Includes bibliographical references and index.
%\    ISBN 0-471-48348-6 (pbk.)
%\    1. Surveys---Methodology.  2. Social 
%\  sciences---Research---Statistical methods.  I. Groves, Robert M.  II. %
%Series.\\
%
%HA31.2.S873 2007
%001.4'33---dc22                                             2004044064
\end{copyrightpage}

\dedication{`Jika Kamu tidak dapat menahan lelahnya belajar, 
Maka kamu harus sanggup menahan perihnya Kebodohan.'
~Imam Syafi'i~}

\begin{contributors}
\name{Rolly Maulana Awangga,} Informatics Research Center., Politeknik Pos Indonesia, Bandung,
Indonesia



\end{contributors}

\contentsinbrief
\tableofcontents
\listoffigures
\listoftables
\lstlistoflistings


\begin{foreword}
Sepatah kata dari Kaprodi, Kabag Kemahasiswaan dan Mahasiswa
\end{foreword}

\begin{preface}
Buku ini diciptakan bagi yang awam dengan git sekalipun.

\prefaceauthor{R. M. Awangga}
\where{Bandung, Jawa Barat\\
Februari, 2019}
\end{preface}


\begin{acknowledgments}
Terima kasih atas semua masukan dari para mahasiswa agar bisa membuat buku ini 
lebih baik dan lebih mudah dimengerti.

Terima kasih ini juga ditujukan khusus untuk team IRC yang 
telah fokus untuk belajar dan memahami bagaimana buku ini mendampingi proses 
Intership.
\authorinitials{R. M. A.}
\end{acknowledgments}

\begin{acronyms}
\acro{ACGIH}{American Conference of Governmental Industrial Hygienists}
\acro{AEC}{Atomic Energy Commission}
\acro{OSHA}{Occupational Health and Safety Commission}
\acro{SAMA}{Scientific Apparatus Makers Association}
\end{acronyms}

\begin{glossary}
\term{git}Merupakan manajemen sumber kode yang dibuat oleh linus torvald.

\term{bash}Merupakan bahasa sistem operasi berbasiskan *NIX.

\term{linux}Sistem operasi berbasis sumber kode terbuka yang dibuat oleh Linus Torvald
\end{glossary}

\begin{symbols}
\term{A}Amplitude

\term{\hbox{\&}}Propositional logic symbol 

\term{a}Filter Coefficient

\bigskip

\term{\mathcal{B}}Number of Beats
\end{symbols}

\begin{introduction}

%% optional, but if you want to list author:

\introauthor{Rolly Maulana Awangga, S.T., M.T.}
{Informatics Research Center\\
Bandung, Jawa Barat, Indonesia}

Pada era disruptif  \index{disruptif}\index{disruptif!modern} 
saat ini. git merupakan sebuah kebutuhan dalam sebuah organisasi pengembangan perangkat lunak.
Buku ini diharapkan bisa menjadi penghantar para programmer, analis, IT Operation dan Project Manajer.
Dalam melakukan implementasi git pada diri dan organisasinya.

Rumusnya cuman sebagai contoh aja biar keren\cite{awangga2018sampeu}.

\begin{equation}
ABC {\cal DEF} \alpha\beta\Gamma\Delta\sum^{abc}_{def}
\end{equation}

\end{introduction}

%%%%%%%%%%%%%%%%%%Isi Buku_

\chapter{Judul Bagian Pertama}
\section{Irvan Rizkiansyah}

\section{Python}
	\subsection{Background}
	\label{Background}
	\par
	Python adalah sebuah bahasa pemrograman yang bersifat interpreter, interactive, object-oriented, dan dapat beroperasi hampir pada semua platform seperti Windows, Linux, Mac. Python termasuk sebagai bahasa pemrograman yang dapat dengan mudah di pelajari karena sintaks yang jelas dan mudah dipahami, dan dapat dikombinasikan dengan penggunaan modul yang siap pakai, dan struktur data tingkat tinggi yang efisien \cite{prasetya2012deteksi}.
	\par
	Python memiliki kepustakaan atau biasa disebut library yang sangat luas, dan dalam distribusi Python yang telah disediakan, hal tersebut diakibatkan oleh pendistribusian Python yang bebas karena bahasa pemrograman Python merupakan bahasa pemrograman yang freeware atau bebas dalam hal pengembangannya. Python adalah sebuah bahasa pemrograman yang dapat dengan mudah dibaca dan terstruktur, hal tersebut dikarenakan penggunaan sistem identasi, yaitu pemisahan blok-blok program susunan identasi, jadi untuk menambahkan sub-sub program dalam sebuah blok program, sub program tersebut harus diletakkan pada satu atau lebih spasi dari kolom sebuah blok \cite{perkasa2014rancang}.
	\par
	Bahasa pemrograman Python dibuat oleh Guido Van Rossum. Dikarenakan para pengembang software atau perangkat lunak lebih cenderung memilih kecepatan dalam menyelesaikan suatu proyek dibandingkan dengan kecepatam proses dari program yang dijalankan, maka dari itu bahasa pemrograman Python dapat dibilang bahasa pemerograman yang kecepatannya dapat melebihi bahasa pemrograman C. Akan tetapi bahasa pemrograman Python lebih lambat dalam memproses suatu program dibandingkan bahasa pemrograman C. dengan berkembangnya kecepatan prosesor dan memori saat ini, mengakibatkan tidak terlihatnya keterlambatan dari sebuah program yang menggunakan bahasa pemrograman Python \cite{miftakhuddinimplementasi}.

	\subsection{Problems}
		\begin{itemize}
			\item Kurangnya pemahaman tentang bahasa pemrograman Python
			\item Kurang mengerti dalam hal fungsi-fungsi yang terdapat pada bahasa pemrograman Python
		\end{itemize}

	\subsection{Objective and Contribution}
		\subsubsection{Objective}
			\begin{itemize}
				\item Dapat memahami tentang bahasa pemrograman Python
				\item Dapat memahami fungsi fungsi yang terdapat pada bahasa pemrograman Python
			\end{itemize}
	
		\subsubsection{Contribution}
			\begin{itemize}
				\item Dapat membangun sebuah sistem dengan menggunakan bahasa pemrograman Python
				\item Dapat membangun sebuah alat yang berguna, menggunakan mikrokontroler dan bahasa pemrograman python
			\end{itemize}

	\subsection{Scoop and Environtment}
		\begin{itemize}
			\item Pengenanalan tentang bahasa pemrograman Python
			\item Pengenalan fungsi-fungsi yang terdapat pada bahasa pemrograman Python
		\end{itemize}

\section{Luthfi Muhammad Nabil\_1174035}
\subsection{Background}
Python adalah sebuah bahasa pemrograman dengan level tinggi yang interaktif, dan mendukung berbagai paradigma pemrograman. Python sudah terkenal pada kalangan programmer sebagai bahasa yang mudah dipahami dan memiliki kompleksitas yang dinamis sehingga dapat dipakai di algoritma maupun platform yang berbagai macam.Python sudah memiliki banyak komunitas pendukung karena penggunanya yang banyak. Selain pada komunitas biasa, Python sudah diimplementasikan pada banyak perusahaan ternama dan dipasang pada aplikasi yang sudah terkenal seperti pada search engine google yang dimiliki oleh perusahaan Google. 
\linebreak
\linebreak
Python mulai dirilis pada tahun 1991 oleh Guido van Rossum sebagai kelanjutan dari bahasa pemrograman ABC dengan memiliki versi yaitu 0.9.0. Nama dari bahasa Python diambil dari program televisi di Inggris bernama Monty Python. Lalu tahun 1995, Guido pindah ke CNRI di Virginia, Amerika sembari melanjutkan pengembangan Python. Versi terakhir yang dikeluarkan telah mencapai 1.6. Pada awalnya, Python adalah bahasa yang dipakai untuk  Lalu pada tahun 2000, dirilis Python versi 2.0 yang memiliki peran sebagai bahasa pemrograman tidak berbayar atau open source. Van Rossum sendiri aktif pada development dari Python tetapi sudah bergabung dengan banyak penyumbang. Dibandingkan dengan bahasa lain, Python sudah melewati beberapa versi yang terbatas, mengikuti filosofi dari perubahan berurutan. 
\linebreak
\linebreak
Untuk memahami bahasa Python tidak sulit, tetapi instalasi Python cukup memiliki trik tersendiri terlebih untuk pengguna yang baru memasuki lingkup programming. Pada sistem operasi windows, pengguna diharuskan untuk memasuki sistem pada windows untuk mengatur lokasi dari Python yang sudah diinstall. Selain itu, untuk yang terbiasa dengan beberapa pemrograman harus beradaptasi dengan aturan - aturan pada bahasa pemrograman Python seperti penggantian titik koma (;) dengan indentasi. Oleh karena itu, penulis akan membahas mengenai pengenalan singkat mengenai bahasa pemrograman python dan cara instalasi dari python dan library pip.

\subsection{Problems}
Sesuai dengan latar belakang yang telah dibahas, penulis merumuskan masalah sebagai berikut : 
\begin{enumerate}	
	\item Bagaimana pemaparan singkat mengenai Python?
	\item Bagaimana cara melakukan instalasi Python?
\end{enumerate}

\subsection{Objective and Contribution}
\subsubsection{Objective}
\begin{enumerate}
	\item Untuk membahas mengenai Python.
	\item Untuk menunjukkan cara instalasi Python.
\end{enumerate}

\subsubsection{Contribution}
Pada materi ini, penulis menggunakan Python.

\subsection{Scoop and Environment}
\begin{itemize}
	\item Pada Chapter 1 membahas mengenai sejarah, latar belakang, dan keterangan singkat mengenai python tersebut. Chapter ini juga merangkum masalah dan mencari tujuan yang ingin dicapai penulis dalam membuat resume ini.
\end{itemize}

\section{Hagan Rowlenstino/1174040}
\subsection{Background}
Python di desain sebagai bahasa pemrograman yang dapat digunakan sehari-hari. Pencipta python ,Guido van Rossum, telah menulis seri lengkap tentang sejarah bahasa tersebut.Python diciptakan di awal 1990 di CWI \'(the Centrum voor Wiskunde and Informatica), tempat kelahiran ALGOL \'(Algorithmic Language 68 ). Sebelumnya, Rossum juga telah mengerjakan bahasa pemrograman ABC, yang dikembangkan di  CWI sebagai bahasa pengajaran yang menekankan kejelasan. Walaupun project ABC telah di tutup , Rossum banyak belajar dari hal tersebut saat dia mulai membuat Python sebagai alat untuk multimedia dan project penelitian sistem operasi. Dia ingin Python mempunyai tingkatan yang cukup tinggi agar mudah untuk dibaca dan ditulis, juga mirip dengan Java, dan menawarkan portabilitas serta error model yang terdefinisi dengan baik.
\linebreak
\linebreak
Python juga kaya akan vocabulary yang berguna untuk membuat algoritma yang kompleks dengan efisien dikarenakan punya dictionaries yang memiliki string yang kuat dan assosiasi array yang fleksibel. Python menggabungkan antara fleksibilitas tingkat tinggi, kemampuan membaca, dan interface yang terdefinisi dengan baik. Kombinasi tersebut membuat Python cocok untuk menyelesaikan masalah komputasi non-algoritma seperti integrase dengan web, format data, ataw hardware kelas rendah. Python mudah untuk dipelajari karena strukturnya sederhana dan sintaksnya jelas, punya library yang portable dan dapat digunakan di beda perangkat,dan dapat terintegrasi dengan bahasa pemrograman lain seperti C, C++, dan Java.
\subsection{Problems}
\begin{enumerate}
\item Banyak pemrograman yang penggunaannya kompleks
\end{enumerate}
\subsection{Objective and Contribution}
\subsubsection{Objective}
\begin{enumerate}
\item Dapat memudahkan pemrograman dengan bahasa pemrograman yang tepat
\end{enumerate}
\subsubsection{Contribution}
\begin{enumerate}
\item Menggunakan Python sebagai bahasa pemrograman
\end{enumerate}
\subsection{Scoop and Environment}
\begin{enumerate}
\item Mengimplementasikan Python dalam pemrograman
\end{enumerate}


\section{Faisal Najib Abdullah 1174042}
\subsection{Background}
\label{Background}
\par
Python lahir pada akhir tahun 1980-an dan implementasinya dimulai pada Desember 1989 oleh Guido van Rossum di CWI di Belanda sebagai penerus bahasa ABC (itu sendiri terinspirasi oleh SETL) yang mampu menangani pengecualian dan berinteraksi dengan sistem operasi Amuba. Van Rossum adalah penulis utama Python, dan peran sentralnya yang berkelanjutan dalam menentukan arah Python tercermin dalam judul yang diberikan kepadanya oleh komunitas Python, Benevolent Dictator for Life (BDFL).
\par
Python adalah bahasa pemrograman interpretatif multiguna dengan filosofi desain yang berfokus pada keterbacaan kode dan python sendiri diklaim sebagai bahasa yang menggabungkan kapabilitas, kemampuan, dengan kode sintaksis yang sangat jelas, dan dilengkapi dengan fungsi pustaka standar yang besar dan komprehensif. Python juga didukung oleh komunitas besar.
\par
Python mendukung pemrograman multi-paradigma, terutama tetapi tidak terbatas pada pemrograman berorientasi objek, pemrograman imperatif, dan pemrograman fungsional. Salah satu fitur yang tersedia di Python adalah sebagai bahasa pemrograman dinamis yang dilengkapi dengan manajemen memori otomatis. Python menggunakan bahasa bahasa scripting yang sama seperti bahasa pemrograman dinamis, meskipun dalam praktiknya penggunaan bahasa ini lebih luas mencakup konteks penggunaan yang umumnya tidak dilakukan menggunakan bahasa skrip. Python dapat digunakan untuk keperluan pengembangan perangkat lunak dan dapat berjalan di berbagai platform sistem operasi.
\par
CPython, implementasi referensi Python, adalah perangkat lunak bebas dan open source dan memiliki model pengembangan berbasis komunitas, seperti halnya hampir semua implementasi alternatifnya. CPython dikelola oleh Yayasan Perangkat Lunak Python nirlaba \cite{van2007python}.

\subsection{Problems}
\begin{itemize}
	\item Mahasiswa D4 TI belum dapat belum memahami apa itu python
    \item Mahasiswa D4 TI belum mengerti fungsi fungsi apa saja yang terdapat pada python
    \item Mahasiswa D4 TI belum dapat menjalankan fungsi python
\end{itemize}

\subsection{Objective and Contribution}
\subsubsection{Objective}
\begin{itemize}
	\item Mahasiswa D4 TI dapat memahami apa itu python
	\item Mahasiswa D4 TI dapat memahami fungsi fungsi yang terdapat pada python
	\item Mahasiswa D4 TI dapat menjalankan fungsi python
\end{itemize}
	
\subsubsection{Contribution}
\begin{itemize}
	\item Mahasiswa D4 TI dapat membangun suatu aplikasi yang mengimplementasikan bahasa python
	\item Mahasiswa D4 TI dapat membangun alat yang terhubung dengan aplikasi menggunakan bahasa python
\end{itemize}

\subsection{Scoop and Environtment}
\begin{itemize}
	\item Mengenali apa itu python pada mahasiswa
	\item Mengenali fungsi fungsi dasar python dan menjalankannya
\end{itemize}


\section{Ichsan Hizman Hardy/1174034}
\subsection{Background}
\par
Python merupakan bahasa pemrograman interpretatif multiguna. Python pertama kali diciptakan oleh Guido van Rossum di Stichting Mathematisch Centrum atau CWI di Belanda pada tahun 1990. pada tahun 1995, Guido melanjutkan karyanya pada Python di Virginia, dimana ia telah meliris beberapa versi perangkat lunak\cite{priyahita2015analisis}.
Tidak seperti bahasa lain yang sulit dibaca dan dipahami, python menekankan keterbacaan kode untuk membuatnya lebih mudah untuk memahami sintaksis\cite{cokelaer2013bioservices}.Ini membuat Python sangat mudah dipelajari untuk pemula dan mereka yang telah menguasai bahasa pemrograman lain.
\par
Python dengan desian yang sangat mudah di baca dan dipahami, karena sama seperti bahasa pemrograman yang lainnya yaitu dengan menggunakan bahasa inggris. selain itu juga lebih sedikit dalam penggunaan rumus atau syntac\cite{nur2018prototipe}.
\par
Pyton juga mendukung sistem teknik pemrograman yang merangkum kode dalam objek. Bahasa Python  mendukung hampir  semua sistem operasi, termasuk operasi Linux\cite{muzawi2018penerapan}.
\par
Dengan kode yang simpel dan mudah diimplementasikan, seorang programer dapat lebih mengutamakan pengembangan aplikasi yang dibuat.
	
\subsection{Problems}
\begin{enumerate}
	\item Mahasiswa D4TI2B belum bisa menggunakan bahasa python.
	\item Bagaimana pengaruh bahasa python terhadap mahasiswa D4TI2B.
	\item Bagaimana penggunaan bahasa python terhadap web service.
\end{enumerate}
	
\subsection{Objective and Contribution}
\subsubsection{Objective}
\begin{enumerate}
	\item Mahasiswa D4TI2B mampu memahami bahasa pemrograman python secara bertahap.
	\item Bahasa pemrograman python mampu mempengaruhi mahasiswa D4TI2B menjadi lebih semangat dalam belajar web service.
	\item Penggunaan bahasa python mampu mempermudah mahasiswa dalam membuat web service.
\end{enumerate}
\subsubsection{Contribution}
\begin{enumerate}
	\item Membantu mahasiswa D4TI2B dalam menyelesaikan masalah pada python.
	\item Membantu mahasiswa D4TI2B memahami bahasa pemrograman python.
	\item Mempelajari bahasa python dalam proses pembuatan web service.
\end{enumerate}
		
\subsection{Scope and Environtment}
\begin{enumerate}
	\item Mahasiswa D4TI2B memahami bahasa pemrograman python.
	\item Mahasiswa D4TI2B mampu menjalankan fungsi python.
	\item Mahasiswa D4TI2B mampu membuat web service menggunakan python.
\end{enumerate}


\chapter{Judul Bagian Kedua}

\section{IrvanRizkiansyah/1174043}
	\subsection{Teori}
		\begin{enumerate}
			\item Pada python variabel tidak perlu dideklarasikan, pendeklarasian terjadi secara otomatis pada saat memberikan suatu nilai atau data ke variabel. Terdapat beberapa jenis tipe data variabel pada python, diantaranya :
				\begin{itemize}
					\item Python Numbers, dimana akan menyimpan data yang berupa angka. Penggunaan pada python sebagai berikut : 
					var1 = 5
					var2 = 48.9
					
					\item Python Text, dimana akan menyimpan data yang berupa teks ataupun karakter. Penggunaan pada python harus diapitkan oleh tanda petik ("..."), contohnya :
					nama = "Irvan"
					jnskelamin = "L"
					
					\item Python Boolean, dimana yang hanya memiliki 2 nilai yaitu True dan False saja. penggunaan pada python huruf pertama harus kapital, contohnya :
					var3 = True
					var4 = False
				\end{itemize}

			\item \begin{itemize}
					\item Meminta input pada user
					nama = input("Masukkan Nama Anda : ")
					
					\item menampilkan output
					print "Hello Nama Saya Adalah",nama
				\end{itemize}

			\item \begin{itemize}
					\item Operator tambah
					a = b + c
					
					\item Operator kurang
					a = b - c
					
					\item Operator kali
					a = b * c
					
					\item Operator bagi
					a = b / c
					
					\item Konversi integer ke string
					konvVar = str(var1)
					
					\item Konversi string ke integer
					konvVar = int(var2)
				\end{itemize}

			\item \begin{itemize}
				\item Pengulangan for, kemampuan mengulang proses data menggunakan urutan apapun, seperti list.
				contoh penggunaan pada Python dan contoh kode adalah :

					\begin{verbatim}
					for i in range(10):
						print(i)
					\end{verbatim}
					
				\item Pengulangan while, kemampuan mengulang proses data yang akan terus berlanjut jika kondisinya True.
				contoh penggunaan pada Python dan contoh kode adalah :
					\begin{verbatim}
					i= 0
					while i < 10 :
						i=i+1
						print ("loop ke =", i)
					\end{verbatim}
				\end{itemize}
				
			\item Pengambilan keputusan berguna untuk menentukan tindakan apa yang akan diambil sesuai dengan kondisi yang ada. Contohnya :
				\begin{verbatim}
				nilai = 9
				if(nilai > 7):
					print("Selamat Anda Lulus")
				else:
					print("Maaf Anda Tidak Lulus")
				\end{verbatim}
				
				Dan untuk kondisi di dalam kondisi contohnya :
				
				\begin{verbatim}
				gaji = 10000000
				berkeluarga = True
				if gaji > 3000000:
					print "Gaji sudah diatas UMR"
					if berkeluarga:
							print "Wajib ikutan asuransi dan menabung untuk pensiun"
						else:
							print "Tidak perlu ikutan asuransi"
				else:
					print "Gaji belum UMR"
				\end{verbatim}

			\item \begin{itemize}
					\item Syntax Errors, Salahnya dalam penulisan sintaks.
					cara penanganannya adalah dengan menganalisa bagian kode yang error dan memperbaiki sintaks tersebut.
					
					\item Exceptions, error yang terjadi karena sintaks tidak dapat dieksekusi.
					cara penanganannya adalah dengan menganalisa bagian kode yang error dan memperbaiki sintaks tersebut.
				\end{itemize}
			
			\item Try Except adalah cara penanganan error pada Python.
			Contohnya : 
				\begin{verbatim}
				x = 0
				try:
					x = 1 / 0
				except Exception, e:
					print e
				\end{verbatim}

		\end{enumerate}
		
	\subsection{Keterampilan Pemrograman}
		\begin{enumerate}
			\item \lstinputlisting{src/chapter2/1174043_1.py}

			\item \lstinputlisting{src/chapter2/1174043_2.py}

			\item \lstinputlisting{src/chapter2/1174043_3.py}

			\item \lstinputlisting{src/chapter2/1174043_4.py}

			\item \lstinputlisting{src/chapter2/1174043_5.py}

			\item \lstinputlisting{src/chapter2/1174043_6.py}

			\item \lstinputlisting{src/chapter2/1174043_7.py}

			\item \lstinputlisting{src/chapter2/1174043_8.py}

			\item \lstinputlisting{src/chapter2/1174043_9.py}

			\item \lstinputlisting{src/chapter2/1174043_10.py}
			
			\item \lstinputlisting{src/chapter2/1174043_11.py}
		\end{enumerate}
		
	\subsection{Keterampilan Penanganan Error}
		\begin{enumerate}
			\item TypeError yaitu error di dalam tipe data disaat melakukan substring dan ingin memasukkannya ke dalam kondisi for 
			yang hanya menerima tipe int. jadi harus merubah tipe inputan yaitu string menjadi integer.

			\item \lstinputlisting{src/chapter2/1174043_2err.py}
		\end{enumerate}

\section{Hagan Rowlenstino/1174040}
\subsection{Teori}
\begin{enumerate}
	\item tipe data teks : ada string yaitu kumpulan karakter dan char adalah karakter. penulisannya harus diapit dengan tanda petik 1,2, ataupun 3
   ('..'), (".."), ('''...'''), ("""...""")

   tipe data angka : ada float yaitu bilangan pecahan dan integer yaitu bilangan bulat. penulisannya yaitu dengan menginisialisasikan nama
   variable lalu masukkan angka (x = 30)

   tipe data boolean : tipe yang memiliki dua nilai yaitu true dan false. penggunaannya huruf pertamanya harus kapital True dan False.

   \item input().inisialisasikan input tersebut x = input() lalu print(x)

   \item +,*,-,/. misal a = '10' maka integerr = int(a) dan misal a= 10 maka stringg = string(a)

   \item while : untuk perulangan yang tidak pasti

   \begin{verbatim}

  i = 0
	while True:
    if i < 10:
        print "Saat ini i bernilai: ", i
        i = i + 1
    elif i >= 10:
        break
   
   for : untuk perulangan yang pasti
	for i in range(0, 10):
    print i
    \end{verbatim}
    \item 
    \begin{verbatim}
    if kondisi:
	hasil

   dan
   if kondisi:
	hasil
	if kondisi:
	    hasil
	\end{verbatim}
	\item type error = ubah tipe str jadi int

	\item taruh try : diatas sintaks yang ingin diketahui jika terjadi error lalu enter dan tulis except: lalu tenkan enter 
dan masukkan tulisaan yang akan ditampilkan.
	\begin{verbatim}
	a = 2
	b = 'as'
	try:
    	print(a + b)
	except TypeError:
    	print("Integer dan String Tidak Dapat
    	 Dijumlah Karena Berbeda Tipe")
	\end{verbatim}

\end{enumerate}
\subsection{Keterampilan Pemrograman}
\begin{enumerate}
	\item \lstinputlisting{src/chapter2/1174040_1.py}

	\item \lstinputlisting{src/chapter2/1174040_2.py}

	\item \lstinputlisting{src/chapter2/1174040_3.py}

	\item \lstinputlisting{src/chapter2/1174040_4.py}

	\item \lstinputlisting{src/chapter2/1174040_5.py}

	\item \lstinputlisting{src/chapter2/1174040_6.py}

	\item \lstinputlisting{src/chapter2/1174040_7.py}

	\item \lstinputlisting{src/chapter2/1174040_8.py}

	\item \lstinputlisting{src/chapter2/1174040_9.py}

	\item \lstinputlisting{src/chapter2/1174040_10.py}
	
	\item \lstinputlisting{src/chapter2/1174040_11.py}
\end{enumerate}
\subsection{Keterampilan Penanganan Error}
\begin{enumerate}
	\item TypeError yaitu error di dalam tipe data disaat melakukan substring dan ingin memasukkannya ke dalam kondisi for 
	yang hanya menerima tipe int. jadi harus merubah tipe inputan yaitu string menjadi integer.

	\item \lstinputlisting{src/chapter2/1174040_2err.py}
\end{enumerate}

 \section{Muhammad Iqbal Panggabean}
\subsection{Teori}
\begin{enumerate}
    \item Jenis jenis variable phyton dan cara pemakaiannya
Variabel merupakan tempat menyimpan data. Dalam Phyton terdapat beberapa variabel dengan berbagai type data diantaranya adalah variabel dengan type data number, string, dan boolean. Dalam phyton kita dapat membuat variable dengan cara sebagai gambar berikut
   \lstinputlisting[firstline=8, lastline=12]{src/1174063_teori.py}
    \item Kode untuk meminta input dari user dan bagaimana melakukan output ke layar
 \lstinputlisting[firstline=67, lastline=68]{src/1174063_teori.py}
    \item Operator dasar aritmatika
Ada operator penambahan, pengurangan perkalian, perkalian, pembagian, modulus, perpangkatan, dan pembulatan decimal.
\lstinputlisting[firstline=71, lastline=94]{src/1174063_teori.py}
    \item Perulangan
Terdapat dua jenis perulangan di dalam phyton yaitu perulangan while dan perulangan for
 \lstinputlisting[firstline=97, lastline=99]{src/1174063_teori.py}
 \lstinputlisting[firstline=102, lastline=105]{src/1174063_teori.py}
    \item sintak Untuk memilih kondisi, dan kondisi didalam kondisi
Pengambilan kondisi If yang digunakan untuk mengantisipasi kondisi yang terjadi saat program dijalankan dan menentukan tindakan apa yang akan diambil sesuai dengan kondisi.
  \lstinputlisting[firstline=108, lastline=111]{src/1174063_teori.py}
  \lstinputlisting[firstline=114, lastline=119]{src/1174063_teori.py}
  \lstinputlisting[firstline=122, lastline=129]{src/1174063_teori.py}

    \item Jenis-jenis error pada phyton
Syntax Errors adalah keadaan dimana kode python mengalami kesalahan penulisan. 
ZeroDivisonError adalah eror yang terjadi saat eksekusi program menghasilkan perhitungan matematika pembagian dengan angka nol.
NameError adalah eror yang terjadi saat kode di eksekusi terhadap local name atau global name yang tidak terdefinisi. 
TypeError adalah eror yang terjadi saat dilakukan eksekusi pada suatu operasi atau fungsi dengan type object yang tidak sesuai.

    \item Cara memakai try except
Cara pemakaian try except adalah sebagai berikut :
\lstinputlisting[firstline=132, lastline=138]{src/1174063_teori.py}

\end{enumerate}

\subsection{praktek}
\begin{enumerate}
    \item Jawaban soal no 1
    \lstinputlisting[firstline=11, lastline=20]{src/1174063_praktek.py}
    \item Jawaban soal no 2
    \lstinputlisting[firstline=24, lastline=28]{src/1174063_praktek.py}
    \item Jawaban soal no 3
    \lstinputlisting[firstline=33, lastline=37]{src/1174063_praktek.py}
    \item Jawaban soal no 4
    \lstinputlisting[firstline=40, lastline=41]{src/1174063_praktek.py}
    \item Jawaban soal no 5
    \lstinputlisting[firstline=44, lastline=56]{src/1174063_praktek.py}
    \item Jawaban soal no 6
    \lstinputlisting[firstline=59, lastline=60]{src/1174063_praktek.py}
    \item Jawaban soal no 7
    \lstinputlisting[firstline=63, lastline=64]{src/1174063_praktek.py}
    \item Jawaban soal no 8
    \lstinputlisting[firstline=67, lastline=71]{src/1174063_praktek.py}
    \item Jawaban soal no 9
    \lstinputlisting[firstline=74, lastline=74]{src/1174063_praktek.py}
    \item Jawaban soal no 10
    \lstinputlisting[firstline=77, lastline=77]{src/1174063_praktek.py}
    \item Jawaban soal no 11
    \lstinputlisting[firstline=80, lastline=80]{src/1174063_praktek.py}
\end{enumerate}

\subsection{Keterampilan dan penanganan eror}
    \lstinputlisting[firstline=10, lastline=17]{src/errr2.py}



\section{Luthfi M. Nabil/1174035}
\subsection{Teori}
\begin{enumerate}
	\item Berikut merupakan jenis - jenis variabel yang terdapat pada python : \begin{itemize}	
	\item Jenis variabel Teks (String) : Merupakan jenis variabel untuk menampung karakter. Cara penulisannya harus diapit dengan tanda petik 1 atau 2 ('..'), ("..")

   \item Jenis variabel numeric(Integer, Float) : Jenis variabel ini menampung nilai berupa angka diantaranya bilangan bulat (integer) dan bilangan koma (float)  Penulisannya yaitu dengan menginisialisasikan nama
   variable lalu masukkan angka (x = 30, x=3.3)

   \item Jenis variabel pengkondisian : tipe yang memiliki dua nilai yaitu true dan false. penggunaannya huruf pertamanya harus kapital True dan False.
   \end{itemize}
   \item input().inisialisasikan input tersebut x = input() lalu print(x)

   \item Operator dasar aritmatika dan mengubah string ke integer dan integer ke string : 
\begin{itemize}
\item Jenis - jenis operator aritmatika : Penjumlahan (+),Perkalian (*), Pengurangan(-),Pembagian(/).
\item Convert int to string dan sebaliknya : misal a = '10' maka integer = int(a) dan misal a= 10 maka string = string(a)
\end{itemize}

   \item While : untuk perulangan yang memiliki kondisi lebih bebas/tidak terpaku

   \begin{verbatim}

     	i = 20
	while True:
        		print "Saat ini i bernilai: ", i
        		i = i - 1
   
   for : untuk perulangan yang pasti
	for i in range(0, 10):
    		print i
    \end{verbatim}
    \item 
    \begin{verbatim}
    if kondisi:
	hasil
   dan
   if kondisi:
	hasil
	if kondisi:
	    hasil
	\end{verbatim}
	\item type error = ubah tipe str jadi int, index error = array index tidak diketahui

	\item taruh try : diatas sintaks yang ingin diketahui jika terjadi error lalu enter dan tulis except: lalu tekan enter 
dan masukkan tulisaan yang akan ditampilkan.
	\begin{verbatim}
	a = 2
	b = 'Coba Coba'
	try:
    	print(a + b)
	except TypeError:
    	print("Integer dan String Tidak Dapat
    	 Dijumlah Karena Berbeda Tipe")
	\end{verbatim}

\end{enumerate}
\subsection{Keterampilan Pemrograman}
\begin{enumerate}
	\item \lstinputlisting{src/chapter2/1174035_1.py}

	\item \lstinputlisting{src/chapter2/1174035_2.py}

	\item \lstinputlisting{src/chapter2/1174035_3.py}

	\item \lstinputlisting{src/chapter2/1174035_4.py}

	\item \lstinputlisting{src/chapter2/1174035_5.py}

	\item \lstinputlisting{src/chapter2/1174035_6.py}

	\item \lstinputlisting{src/chapter2/1174035_7.py}

	\item \lstinputlisting{src/chapter2/1174035_8.py}

	\item \lstinputlisting{src/chapter2/1174035_9.py}

	\item \lstinputlisting{src/chapter2/1174035_10.py}
	
	\item \lstinputlisting{src/chapter2/1174035_11.py}
\end{enumerate}
\subsection{Keterampilan Penanganan Error}
\begin{enumerate}
	\item \begin{itemize} 
		\item TypeError yaitu error di dalam variabel disaat melakukan substring dan ingin memasukkannya ke dalam kondisi for 
	yang hanya menerima tipe int. jadi harus merubah tipe inputan yaitu string menjadi integer.
		\item IndexError yaitu error saat array dengan index yang telah dipilih tidak ditemukan atau tidak memiliki nilai
		\end{itemize}

	\item \lstinputlisting{src/chapter2/1174035_2err.py}
\end{enumerate}

\section{Faisal Najib Abdullah 1174042}
\subsection{Teori}
\begin{enumerate}
	\item Variabel merupakan tempat menyimpan data, sedangkan tipe data adalah jenis data yang terseimpan dalam variabel.
	\lstinputlisting{src/chapter2/1174042_1,1.py}
	Variabel x memiliki nilai aku, variable y memiliki nilai sayang, dan variabel z memiliki nilai najib. karna memiliki type data string maka kata kata tersebut jika di tambahkan berubah menjadi sebuah kalimat
	
	\item Input, untuk membuat kode input, pertama buat variabel x yang berisi input seperti pada contoh jika di run maka langsung diminta untuk memasukan NIM ketika di enter hasilnya berupa Hello, 1174042
	\lstinputlisting{src/chapter2/1174042_1,2.py}
	
	\item Untuk merubah type data dari string ke integer, tambahkan kata int lalu kurung buka lalu nama variabel yang akan dirubah dan kurung tutup seperti pada contoh.
	\lstinputlisting{src/chapter2/1174042_1,3.py}
	
	\item untuk perulangan disini menggunakan while variabel i bernilai 0, kemudian while i lebih kecil dari 6 jika benar maka akan terus dilakukan pengulangan dan jika salah tidak akan dilakukan pengulangan, i selalu bertambah 1, dan menampilkan nilai i. 
	\lstinputlisting{src/chapter2/1174042_1,4.py}
	
	\item membuat 2 variabel a dan b, variabel a bernilai 200 dan b bernilai 33 jika b lebihbesar dari a maka akan menampilkan sesuai perintah seperti contoh.
	\lstinputlisting{src/chapter2/1174042_1,5.py}
	
	\item Jenis error yang sering di alami pada python
	\begin{itemize}
	    \item menjumlahkan bilangan yang berbeda type data. Solosinya rubah dan sesuaikan type data yang dibutuhkan
	    \item sepasi pada kondisi yang harus sejajar. Sejajarkan posisi sesuai kondisi
	    \item Typo. Cek kembali agar tidak terjadi kesalahan code
	\end{itemize}
	
	\item untuk menggunakan try, pertama tuliskan coba terlebih dahulu code apakah terjadi error atau tidak. Jika terjadi error copy TypeError kemudian tuliskan try sebelum line yg error, dibawah line yg error tuliskan except dan paste typeerror yang sebelumnya sudah di copy, kemudian tuliskan kenapa bisa terjadi error menggunakan katakata sendiri.
	\lstinputlisting{src/chapter2/1174042_1,7.py}
\end{enumerate}

\subsection{Keterampilan Pemrograman}
\begin{enumerate}
    \item \lstinputlisting{src/chapter2/1174042_2,1.py}
    
    \item \lstinputlisting{src/chapter2/1174042_2,2.py}
    
    \item \lstinputlisting{src/chapter2/1174042_2,3.py}
    
    \item \lstinputlisting{src/chapter2/1174042_2,4.py}
    
    \item \lstinputlisting{src/chapter2/1174042_2,5.py}
    
    \item \lstinputlisting{src/chapter2/1174042_2,6.py}
    
    \item \lstinputlisting{src/chapter2/1174042_2,7.py}
    
    \item \lstinputlisting{src/chapter2/1174042_2,8.py}
    
    \item \lstinputlisting{src/chapter2/1174042_2,9.py}
    
    \item \lstinputlisting{src/chapter2/1174042_2,10.py}
    
    \item \lstinputlisting{src/chapter2/1174042_2,11.py}
    
\end{enumerate}

\subsection{Keterampilan Penanganan Error}
\begin{enumerate}
    \item Pada saat mengerjakan praktek kedua ini error hanya pada kesalahan type data yaitu TypeError:, solusinya yaitu merumah type data.
    
    \item \lstinputlisting{src/chapter2/1174042_2err.py}
    
\end{enumerate}

\section{Dika Sukma Pradana 1174050}
\subsection{Teori}
\begin{enumerate}
	\item Jenis-jenis variable phyton dan cara pemakaiannya
Variabel merupakan tempat menyimpan data. Dalam Phyton terdapat beberapa variabel dengan berbagai type data diantaranya adalah variabel dengan type data number, string, dan boolean. Dalam phyton kita dapat membuat variable dengan cara sebagai gambar berikut
   \lstinputlisting[firstline=8, lastline=12]{src/1174050_teori.py}
	\item Kode input user dan melakukan output ke layar
 \lstinputlisting[firstline=67, lastline=68]{src/1174050_teori.py}
	\item Operator dasar aritmatika
Macam operator penambahan, pengurangan perkalian, perkalian, pembagian, modulus, perpangkatan, dan pembulatan decimal.
\lstinputlisting[firstline=71, lastline=94]{src/1174050_teori.py}
	\item Perulangan
Macam perulangan di dalam phyton yaitu perulangan while dan perulangan for
 \lstinputlisting[firstline=97, lastline=99]{src/1174050_teori.py}
 \lstinputlisting[firstline=102, lastline=105]{src/1174050_teori.py}
	\item sintak Untuk memilih kondisi, dan kondisi didalam kondisi
Pengambilan kondisi If yang digunakan untuk mengantisipasi kondisi yang terjadi saat program dijalankan dan menentukan tindakan apa yang akan diambil sesuai dengan kondisi.
  \lstinputlisting[firstline=108, lastline=111]{src/1174050_teori.py}
  \lstinputlisting[firstline=114, lastline=119]{src/1174050_teori.py}
  \lstinputlisting[firstline=122, lastline=129]{src/1174050_teori.py}

	\item Jenis-jenis error pada phyton
Syntax Errors adalah keadaan dimana kode python mengalami kesalahan penulisan. 
ZeroDivisonError adalah eror yang terjadi saat eksekusi program menghasilkan perhitungan matematika pembagian dengan angka nol.
NameError adalah eror yang terjadi saat kode di eksekusi terhadap local name atau global name yang tidak terdefinisi. 
TypeError adalah eror yang terjadi saat dilakukan eksekusi pada suatu operasi atau fungsi dengan type object yang tidak sesuai.

	\item Cara memakai try except
Cara pemakaian try except :
\lstinputlisting[firstline=132, lastline=138]{src/1174050_teori.py}

\end{enumerate}

\subsection{Praktek}
\begin{enumerate}
	\item Jawaban soal no 1
	\lstinputlisting[firstline=11, lastline=20]{src/1174050_praktek.py}
	\item Jawaban soal no 2
	\lstinputlisting[firstline=24, lastline=28]{src/1174050_praktek.py}
	\item Jawaban soal no 3
	\lstinputlisting[firstline=33, lastline=37]{src/1174050_praktek.py}
	\item Jawaban soal no 4
	\lstinputlisting[firstline=40, lastline=41]{src/1174050_praktek.py}
	\item Jawaban soal no 5
	\lstinputlisting[firstline=44, lastline=56]{src/1174050_praktek.py}
	\item Jawaban soal no 6
	\lstinputlisting[firstline=59, lastline=60]{src/1174050_praktek.py}
	\item Jawaban soal no 7
	\lstinputlisting[firstline=63, lastline=64]{src/1174050_praktek.py}
	\item Jawaban soal no 8
	\lstinputlisting[firstline=67, lastline=71]{src/1174050_praktek.py}
	\item Jawaban soal no 9
	\lstinputlisting[firstline=74, lastline=74]{src/1174050_praktek.py}
	\item Jawaban soal no 10
	\lstinputlisting[firstline=77, lastline=77]{src/1174050_praktek.py}
	\item Jawaban soal no 11
	\lstinputlisting[firstline=80, lastline=80]{src/1174050_praktek.py}
\end{enumerate}

\subsection{Keterampilan dan Penanganan Eror}
	\lstinputlisting[firstline=10, lastline=17]{src/errord1ka.py}

\section{Rangga Putra Ramdhani}
\subsection{Teori}
\begin{enumerate}
    \item Jenis jenis variable phyton dan cara pemakaiannya
Variabel adalah tempat menyimpan nya objek informasi informasi yang dinamis. Dalam Phyton terdapat beberapa variabel dengan berbagai type data diantaranya adalah variabel dengan type data number, string, dan boolean. Dalam phyton kita dapat membuat variable dengan cara sebagai gambar berikut
   \lstinputlisting[firstline=8, lastline=12]{src/1174056_teori.py}
    \item Kode untuk meminta input dari user dan bagaimana melakukan output ke layar
 \lstinputlisting[firstline=67, lastline=68]{src/1174056_teori.py}
    \item Operator dasar aritmatika
Ada operator penambahan, pengurangan perkalian, perkalian, pembagian, modulus, perpangkatan, dan pembulatan decimal.
\lstinputlisting[firstline=71, lastline=94]{src/1174056_teori.py}
    \item Perulangan
Ada dua jenis perulangan di dalam phyton yaitu perulangan while dan perulangan for
 \lstinputlisting[firstline=97, lastline=99]{src/1174056_teori.py}
 \lstinputlisting[firstline=102, lastline=105]{src/1174056_teori.py}
    \item sintak Untuk memilih kondisi, dan kondisi didalam kondisi
Pengambilan kondisi If yang digunakan untuk mengantisipasi kondisi yang terjadi saat program dijalankan dan menentukan tindakan apa yang akan diambil sesuai dengan kondisi.
  \lstinputlisting[firstline=108, lastline=111]{src/1174056_teori.py}
  \lstinputlisting[firstline=114, lastline=119]{src/1174056_teori.py}
  \lstinputlisting[firstline=122, lastline=129]{src/1174056_teori.py}

    \item Jenis-jenis error pada phyton
Syntax Errors adalah keadaan dimana kode python mengalami kesalahan dalam penulisan. 
ZeroDivisonError adalah eror yang terjadi saat eksekusi program menghasilkan perhitungan matematika pembagian dengan angka nol.
NameError adalah eror yang terjadi saat kode di eksekusi terhadap local name atau global name yang tidak terdefinisi. 
TypeError adalah eror yang terjadi saat dilakukan eksekusi pada suatu operasi atau fungsi dengan type object yang tidak sesuai.

    \item Cara memakai try except
Cara pemakaian try except adalah sebagai berikut :

\lstinputlisting[firstline=132, lastline=138]{src/1174056_teori.py}

\end{enumerate}

\subsection{praktek}
\begin{enumerate}
    \item Jawaban soal no 1
    \lstinputlisting[firstline=11, lastline=20]{src/1174056_praktek.py}
    \item Jawaban soal no 2
    \lstinputlisting[firstline=24, lastline=28]{src/1174056_praktek.py}
    \item Jawaban soal no 3
    \lstinputlisting[firstline=33, lastline=37]{src/1174056_praktek.py}
    \item Jawaban soal no 4
    \lstinputlisting[firstline=40, lastline=41]{src/1174056_praktek.py}
    \item Jawaban soal no 5
    \lstinputlisting[firstline=44, lastline=56]{src/1174056_praktek.py}
    \item Jawaban soal no 6
    \lstinputlisting[firstline=59, lastline=60]{src/1174056_praktek.py}
    \item Jawaban soal no 7
    \lstinputlisting[firstline=63, lastline=64]{src/1174056_praktek.py}
    \item Jawaban soal no 8
    \lstinputlisting[firstline=67, lastline=71]{src/1174056_praktek.py}
    \item Jawaban soal no 9
    \lstinputlisting[firstline=74, lastline=74]{src/1174056_praktek.py}
    \item Jawaban soal no 10
    \lstinputlisting[firstline=77, lastline=77]{src/1174056_praktek.py}
    \item Jawaban soal no 11
    \lstinputlisting[firstline=80, lastline=80]{src/1174056_praktek.py}
\end{enumerate}

\subsection{Keterampilan dan penanganan eror}
    \lstinputlisting[firstline=10, lastline=17]{src/err2.py}

\section{Fathi Rabbani/1164074}
\subsection{Teori}
	\begin{enumerate}
	\item Variables
	\subitem
	Variable yang ada pada Python adalah penggunaan tipe data yaitu ada Variable String,  Integer, dan boolean dimana data yang dimiliki nilainya berupa tipe data String atau Integer, contohnya adalah sebagai berikut
	\item Input and Output
		\lstinputlisting{src/chapter2/1164074/1164074_1.py}
	\item Operator Arithmetic
	\begin{itemize}
	\item Tambah
		\lstinputlisting{src/chapter2/1164074/1164074_2.py}
	\item Kali
		\lstinputlisting{src/chapter2/1164074/1164074_3.py}
	\item Kurang
		\lstinputlisting{src/chapter2/1164074/1164074_4.py}
	\item Bagi
		\lstinputlisting{src/chapter2/1164074/1164074_5.py}
	\item Int to Str
		\lstinputlisting{src/chapter2/1164074/1164074_6.py}
	\item Str to Int
		\lstinputlisting{src/chapter2/1164074/1164074_7.py}
	\end{itemize}

	\item Loop
		\lstinputlisting{src/chapter2/1164074/1164074_8.py}
	\item Kondisi
		\lstinputlisting{src/chapter2/1164074/1164074_9.py}
	\item Error
	\subitem
	Error yang ditemui salah satunya adalah data tipe yang tidak sama namun ingin di lihat hasilnya, seperti contoh berikut :
		\lstinputlisting{src/chapter2/1164074/1164074_10.py}
	cara menanganinya adalah dengan menggunakan data tipe yang sesuai.
	\item Try Except
		\lstinputlisting{src/chapter2/1164074/1164074_11.py}
	\end{enumerate}
\subsection{Praktik}
	\begin{enumerate}
	\item 
	\lstinputlisting{src/chapter2/1164074/1164074_12.py}
	\item
	\lstinputlisting{src/chapter2/1164074/1164074_13.py}
	\item
	\lstinputlisting{src/chapter2/1164074/1164074_14.py}
	\item
	\lstinputlisting{src/chapter2/1164074/1164074_15.py}
	\item
	\lstinputlisting{src/chapter2/1164074/1164074_16.py}
	\item
	\lstinputlisting{src/chapter2/1164074/1164074_17.py}
	\item
	\lstinputlisting{src/chapter2/1164074/1164074_18.py}
	\item
	\lstinputlisting{src/chapter2/1164074/1164074_19.py}
	\item
	\lstinputlisting{src/chapter2/1164074/1164074_20.py}
	\item
	\lstinputlisting{src/chapter2/1164074/1164074_21.py}
	\item
	\lstinputlisting{src/chapter2/1164074/1164074_22.py}
	\end{enumerate}

\subsection{Penanganan Error}
	\begin{enumerate}
	\item
	\subitem
	Dimana error yang saya akan menjadi kan contoh error yang saya dapat pada nomor 5. Dalam no 5 terdapat 7 variabel dan akan di tampilkan output berupa string. Pada awalnya saya membuat source code seperti berikut:
	\begin{verbatim}
	x = str(a,b,c,d,e,f,g)
	print(x)
	\end{verbatim}
	Sehingga output yang diinginkan adalah 1 1 6 4 0 7 4. Untuk menampilkan output String maka source code yang di berikan adalah sebagai berikut :
	\begin{verbatim}
	x = str(a)+str(b)+str(c)+str(d)+str(e)+str(f)+str(g)
	\end{verbatim}
	Output yang di keluarkan nantinya adalah 1164074 dengan type data String

	\item
	\lstinputlisting{src/chapter2/1164074/1164074_2err.py}
	\end{enumerate}


	



\bibliographystyle{IEEEtran} 
%\def\bibfont{\normalsize}
\bibliography{references}


%%%%%%%%%%%%%%%
%%  The default LaTeX Index
%%  Don't need to add any commands before \begin{document}
\printindex

%%%% Making an index
%% 
%% 1. Make index entries, don't leave any spaces so that they
%% will be sorted correctly.
%% 
%% \index{term}
%% \index{term!subterm}
%% \index{term!subterm!subsubterm}
%% 
%% 2. Run LaTeX several times to produce <filename>.idx
%% 
%% 3. On command line, type  makeindx <filename> which
%% will produce <filename>.ind 
%% 
%% 4. Type \printindex to make the index appear in your book.
%% 
%% 5. If you would like to edit <filename>.ind 
%% you may do so. See docs.pdf for more information.
%% 
%%%%%%%%%%%%%%%%%%%%%%%%%%%%%%

%%%%%%%%%%%%%% Making Multiple Indices %%%%%%%%%%%%%%%%
%% 1. 
%% \usepackage{multind}
%% \makeindex{book}
%% \makeindex{authors}
%% \begin{document}
%% 
%% 2.
%% % add index terms to your book, ie,
%% \index{book}{A term to go to the topic index}
%% \index{authors}{Put this author in the author index}
%% 
%% \index{book}{Cows}
%% \index{book}{Cows!Jersey}
%% \index{book}{Cows!Jersey!Brown}
%% 
%% \index{author}{Douglas Adams}
%% \index{author}{Boethius}
%% \index{author}{Mark Twain}
%% 
%% 3. On command line type 
%% makeindex topic 
%% makeindex authors
%% 
%% 4.
%% this is a Wiley command to make the indices print:
%% \multiprintindex{book}{Topic index}
%% \multiprintindex{authors}{Author index}

\end{document}

