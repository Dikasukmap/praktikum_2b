\section{Faisal Najib Abdullah 1174042}
\subsection{Teori}
\begin{enumerate}
	\item Variabel merupakan tempat menyimpan data, sedangkan tipe data adalah jenis data yang terseimpan dalam variabel.
	\lstinputlisting{1174042_1,1.py}
	Variabel x memiliki nilai aku, variable y memiliki nilai sayang, dan variabel z memiliki nilai najib. karna memiliki type data string maka kata kata tersebut jika di tambahkan berubah menjadi sebuah kalimat
	
	\item Input, untuk membuat kode input, pertama buat variabel x yang berisi input seperti pada contoh jika di run maka langsung diminta untuk memasukan NIM ketika di enter hasilnya berupa Hello, 1174042
	\lstinputlisting{1174042_1,2.py}
	
	\item Untuk merubah type data dari string ke integer, tambahkan kata int lalu kurung buka lalu nama variabel yang akan dirubah dan kurung tutup seperti pada contoh.
	\lstinputlisting{1174042_1,3.py}
	
	\item untuk perulangan disini menggunakan while variabel i bernilai 0, kemudian while i lebih kecil dari 6 jika benar maka akan terus dilakukan pengulangan dan jika salah tidak akan dilakukan pengulangan, i selalu bertambah 1, dan menampilkan nilai i. 
	\lstinputlisting{1174042_1,4.py}
	
	\item membuat 2 variabel a dan b, variabel a bernilai 200 dan b bernilai 33 jika b lebihbesar dari a maka akan menampilkan sesuai perintah seperti contoh.
	\lstinputlisting{1174042_1,5.py}
	
	\item Jenis error yang sering di alami pada python
	\begin{itemize}
	    \item menjumlahkan bilangan yang berbeda type data. Solosinya rubah dan sesuaikan type data yang dibutuhkan
	    \item sepasi pada kondisi yang harus sejajar. Sejajarkan posisi sesuai kondisi
	    \item Typo. Cek kembali agar tidak terjadi kesalahan code
	\end{itemize}
	
	\item untuk menggunakan try, pertama tuliskan coba terlebih dahulu code apakah terjadi error atau tidak. Jika terjadi error copy TypeError kemudian tuliskan try sebelum line yg error, dibawah line yg error tuliskan except dan paste typeerror yang sebelumnya sudah di copy, kemudian tuliskan kenapa bisa terjadi error menggunakan katakata sendiri.
	\lstinputlisting{1174042_1,7.py}
\end{enumerate}

\subsection{Keterampilan Pemrograman}
\begin{enumerate}
    \item \lstinputlisting{1174042_2,1.py}
    
    \item \lstinputlisting{1174042_2,2.py}
    
    \item \lstinputlisting{1174042_2,3.py}
    
    \item \lstinputlisting{1174042_2,4.py}
    
    \item \lstinputlisting{1174042_2,5.py}
    
    \item \lstinputlisting{1174042_2,6.py}
    
    \item \lstinputlisting{1174042_2,7.py}
    
    \item \lstinputlisting{1174042_2,8.py}
    
    \item \lstinputlisting{1174042_2,9.py}
    
    \item \lstinputlisting{1174042_2,10.py}
    
    \item
    
\end{enumerate}

\subsection{Keterampilan Penanganan Error}
\begin{enumerate}
    \item Pada saat mengerjakan praktek kedua ini error hanya pada kesalahan type data yaitu TypeError:, solusinya yaitu merumah type data.
    
    \item \lstinputlisting{1174042_2err.py}
    
\end{enumerate}